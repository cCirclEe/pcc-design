\documentclass[a4paper,twoside,BCOR=20mm]{scrreprt}
\usepackage[utf8]{inputenc}
\usepackage[T1]{fontenc}
\usepackage[ngerman]{babel}	% german hyphenation, quotes, etc
\usepackage[ngerman]{translator}
\usepackage{amsmath}
\usepackage{paralist}
\usepackage{amsfonts}
\usepackage{acronym}
\usepackage{enumerate}
\usepackage{hyperref}
\usepackage{amssymb}
\usepackage{caption}
\usepackage{multirow}
\usepackage{graphicx}
\usepackage{tabularx}
\usepackage{color}
\usepackage{wrapfig} % wrap text around figures
\usepackage{subfig} % align two pics beside each other
\usepackage[table,xcdraw]{xcolor}
\hypersetup{ 					% ‘texdoc hyperref‘ for options
	pdftitle={PSE PCC: Entwurf},
	pdfauthor={Giorgio Groß, Christoph Hörtnagl, David Laubenstein,  Josh Romanowski,  Fabian Wenzel},
	bookmarks=true,
}
\title{Entwurf: Privacy Crash- Cam}

%Paket laden
\usepackage[
numberedsection,
nonumberlist, %keine Seitenzahlen anzeigen
acronym,      %ein Abkürzungsverzeichnis erstellen
toc,          %Einträge im Inhaltsverzeichnis
section]      %im Inhaltsverzeichnis auf section-Ebene erscheinen
{glossaries}

%Befehle für Abkürzungen
\newacronym{KIT}{KIT}{Karlsruher Institut für Technologie}

%Richtige Silbentrennungen
\hyphenation{Ein-stel-lun-gen}
\hyphenation{ge-star-tet}

% KIT layout

\definecolor{orange}{rgb}{1,0.5,0}
\definecolor{mintgreen}{RGB}{50,161,137}
\definecolor{gray}{RGB}{120,120,120}

\usepackage[color]{changebar}
\cbcolor{gray}
\changebarwidth 0.5pt

\usepackage{fancyhdr}
\pagestyle{fancy}
 \fancyhf{} %alle Kopf- und Fußzeilenfelder bereinigen 
 
 \fancypagestyle{plain}{} %Kopf- und Fußzeile auf jeder Seite	 
	\fancyhead[L]{Entwurf}
	\fancyhead[R]{\leftmark}
	\rhead{\nouppercase{\leftmark}}
	\renewcommand{\headrulewidth}{0.5pt}
	\renewcommand{\headrule}{\hbox to\headwidth{%
		\color{mintgreen}\leaders\hrule height \headrulewidth\hfill}}

\raggedbottom

	\renewcommand{\footrulewidth}{0.5pt}
	\renewcommand{\footrule}{\hbox to\headwidth{%
  		\color{mintgreen}\leaders\hrule height \headrulewidth\hfill}}				
	\fancyfoot[LE,RO]{\thepage}


\setcounter{tocdepth}{5}
\makeglossaries
\begin{document}
\include{./subtopicsDesign/Deckblatt}
% \maketitle
\tableofcontents
\newpage
%content

% PUT GENERAL (INTRODUCTION ...) CONTENT HERE
\chapter{Architektur}
\section{Model View Presenter (MVP)}
Beim Design der Privacy Crash Cam wird schnell deutlich, dass eine strikte Trennung zwischen Datenbank, Applikationslogik und Benutzeroberfläche von Vorteil ist. Neben einer übersichtlichen Unterteilung in einzelne Module unterstützt dieses Vorgehen die Austauschbarkeit der einzelnen Module, deren Wiederverwendbarkeit und Plattformunabhängigkeit. Das Model-View-Presenter-Prinzip realisiert diese Trennung und wird auf auf App und Web interface übertragen.\linebreak\par

\includegraphics[scale=0.5]{./resources/Diagramme/overview_mvp.jpg}

Die Model-Rolle wird vom Web-Dienst erfüllt. Dieser bietet eine REST API, die sowohl vom Web Interface als auch von der App per HTTP-Anfragen angesprochen wird. Falls weitere Software entwickelt wird, mit der auf die Datenbank zugeriffen werden soll, soll diese ebenfalls auf die selbe API zugrifen können.\linebreak
Darüber hanus besitzt die App ein zusätzliches Model-Modul, das den Speicherzugriff regelt.\linebreak\par

Die Apllikationslogik wird durch die Presenter-Ebene umgesetzt. Hierbei besitzen App und Web Interface eigene Presenter-Module, die an die jeweilige Plattform angepasst sind. Der Presenter handhabt Eingaben durch Nutzer oder Sensoren und koordiniert die darauf folgenden Aktionen, wie das Ändern der Ansicht, Aktualisieren der Ansichten und den Zugriff auf die Model Ebene.\linebreak\par 

Die Rolle der View übernimmt schließlich die GUI. In Android wird diese durch entsprechende XML Dokumente implementiert, in Vaadin durch spezielle Java Klassen die von der Vaadin-Internen Klasse ``UI'' erben. Instanzierung, Manipulation und Navigation erfolgen schließlich durch die Presenter-Ebene, die Bindung erfolgt durch Beobachter und ist bereits durch das jewilige Framework festgelegt.



% PUT APP CONTENT HERE
\chapter{App} \label{chap:app}
\section{Architektur}

\begin{figure}[ht]
	\centering
\includegraphics[width=1\textwidth]{./resources/Diagramme/App/UMLAndroidApp.jpg}
\caption{UML Diagram der Android App}
	\label{fig:modules_overview}
\end{figure}

\subsection{Entwurfsmuster}
In der App werden Entwurfsmuster verwendet, um Komponenten leicht austauschen zu können und um die Nutzeroberfläche und den Funktionsumfang einfach erweitern zu können. Der Einsatz von Entwurfsmustern Verhindert  zudem, dass Klassen einer Schicht Zugriff auf die Klassen der Übernächsten Schicht haben müssen und reduziert die Aufrufe von Klassen einer niederen Schicht auf die darüber liegende Schicht.

\subsubsection{Facade}
Das Facade Muster ist bei der Klasse \nameref{app:klasse:LogInHelper} zu sehen. Sie vereinfacht die Überprüfung, ob eine Nutzer bereits eingeloggt ist und verhindert, dass die View-Schicht direkt auf die Model-Schicht zugreifen muss.

\subsubsection{Observer}
Das Observer Muster reduziert die Aufrufe von Klassen einer niederen Schicht auf die darüber liegende Schicht, wie es das \nameref{app:klasse:IRecordCallback}, das \nameref{app:klasse:IPersistCallback} und das \nameref{app:klasse:IServerResponseCallback} tun. Nützlich ist das Observer Muster vor allem dann, wenn ein asynchroner Task ausgeführt wird und der Aufrufer erst benachrichtigt werden soll, wenn dieser Task zu Ende ist.\newline
Bei der Umsetzung des Observer Musters weicht die App von der herkömmlichen Art ab: Während normalerweise eine Klasse, die als Observer agiert, das Interface, also einen der Callbacks, implementiert, hat in der App im Gegensatz dazu eine Klasse ein Attribut vom Typ des jeweiligen Interfaces. Bei der Instanziierung dieses Attributes findet die Implementierung des Interfaces statt.

\subsubsection{Command}
Das Command Muster kommt bei jedem asynchronen Task zum einsatz. In den Konstruktoren der Tasks wird der Command erstellt, durch die \textit{start} Methode wird der Command ausgeführt. Somit muss sich keine Komponente um die eigentliche Ausführung langwieriger Operationen kümmern.

\subsubsection{Proxy}
Das Proxy Muster kommt zum Einsatz, sobald Anfragen an den Server gesendet werden oder Operationen auf dem Speicher ausgeführt werden sollen. Die entsprechenden Klassen sind \nameref{app:klasse:ServerProxy} und \nameref{app:klasse:MemoryManager}. Dadurch muss keine der Klassen, die den ServerProxy oder den MemoryManager verwenden über die zugrunde liegende Infrastruktur und deren Handhabung bescheid wissen.

\subsubsection{Template Method}
Das Template Method Muster, zu Deutsch ``Schablonenmethode'', findet seinen Einsatz in der \nameref{app:klasse:ContainerActivity}. Die ContainerActivity ist eine Activity, die immer nachd dem gleichen Schema aufgebaut ist: Eine Toolbar am oberen Bildschirmrand und ein Fragment mit dem Inhalt darunter. Sie selbst lädt immer die gleiche Ansicht, lässt aber ihre Unterklasse bestimmen, welches Fragment angezeigt wird. Dadurch können ohne großen Aufwandt witere Ansichten eingefügt werden. Der einzige Aufwandt besteht im erben von ContainerActivity und dem Implementieren der \textit{selectFragment} Methode.\newline
Sollte später eine Activity mit einem ViewPager eingefügt werden, der beispielsweise Aufgenommene und Heruntergeladene Videos in eigenen Tabs anzeigen kann, bietet sich die Schablonenmethode wieder an: In diesem Fall gäbe es eine Acitivty, die ein Layout lädt, welches über einen ViewPager verfügt. Die Unterklassen bestimmen dann auf welcher Seite welches Fragment eingeblendet wird. Natürlich müsste die Nummer des aktuellen Tabs des ViewPagers in der \textit{selectFragment} Methode mitgeliefert werden.

\subsubsection{Strategy}
Die Strategy Methode kommt in den Klassen \nameref{app:klasse:CameraHandler}, \nameref{app:klasse:IFileEncryptor} und \nameref{app:klasse:IKeyEncryptor}  zum Einsatz. Sie ermöglicht die Austauschbarkeit der Kameraimplementierung bzw. der Verschlüsselungsalgorithmen und macht ihre Klienten nur von den Schnittstellen abhängig.

\section{Datenhaltung}
Die App muss sich um die verwaltung von Accountdaten, Einstellungen, Videos, Metadated und symmetrischen Schlüsseln kümmern. Für Accountdaten und  Einstllungen bieten sich die von Android bereit gestllten SharedPreferences an, die als Key-Value-Pairs aufgebaut werden. Das Schema ist in Schaubild \ref{fig:sharedpreferences_overview} veranschaulicht. Der Key entspricht den Daten die abgefragt werden sollen, also Einstellungen oder Account. Die gelesenen Werte bestehen jeweils aus einem String im JSON Format. Dadurch erreicht man eine Bündelung aller zusammengehörender Werte unter einem Key und unterstützt die Änderbarkeit und Ergänzbarkeit bestehender Daten.\newline\par

Möchte man in zukunft Beispielsweise das ``fps''-Feld der Einstellungen entfernen, reicht es, den Konstruktor der Einstellungen-Klasse~\eqref{app:class:settings} das fps-Feld nicht mehr lesen zu lassen und den bisherigen JSON String zu überschreiben. So vermeidet man eine Konvention für die Key-Wahl einführen zu müssen, zum Beispiel ``account.name'' oder ``settings.fps''. Eine solche Vorgehensweise kann auch leicht zu verwirrung Führen, da man aus der selben SharedPreferences-Instanz direkt Daten von zwei nicht zusammenhängenden Typen lesen müsste. Für jeden Typ eigene SharedPreferences anzulegen würde die Lesbarkeit nach dem Hinzufügen weiterer Typen ebenfalls erschweren.

Videos, Metadated und symmetrische Schlüssel können jedoch nicht so einfach in den SharedPreferences abgespeichert werden. Hier gibt es zwei Lösungen: Entweder man speichert sie im internen oder im externen Speicher. Wir werden die erste Variante wählen, wenn auch wir uns die möglichkeit zum Exportieren der Daten in den externen Speicher frei halten werden. Behält man aber eine Referenz auf diese in den den SharedPreferences, hat man keine Chance mehr an die Daten zu kommen, nachdem der Nutzer alle appinterne Daten gelöscht hat.\newline
Aus diesem Grund muss die Abfrage der Videos, Metadaten und Schlüssel ohne verwendung der SharedPreferences erfolgen. Dazu existieren drei Ordner: Ein Video-, ein Metadata- und ein Keyordner. Jedes Video erhält einen eindeutigen Namen, bestehend aus der exakten Aufnahmezeit. Jede mit diesem Video verwandte Datei, also dessen Metadaten und Schlüssel, besitzen den gleichen Dateinamen.\newline

Bezüglich der Metadaten muss auch beachtet werden, dass sie über die App ausgelesen werden können müssen, sie jedoch beim Speichervorgang des Videos direkt verschlüsselt werden. Aus diesem Grund werden die Metadaten in zwei Inhaltlich identischen Dateien abgelegt. Eine der beiden wird veschlüsselt, die Andere nicht. An den Dateinamen der unverschlüsselten Datei wird zur identifikation schließlich ``\_readable'' angehängt\newline\par

Zusätzlich zu den erwähnten Ordnern existiert ein weiterer Ordner, der verwendet wird, um temporäre Videodateien abzulegen. Unter einer temporäre Videodatei ist hier ein unverschlüsseltes Video zu verstehen, welche nur zwischengespeichert wird. Android bietet hierzu die Möglichkeit, die temporäre Datei im appinternen Cache-Ordner der App abzulegen. 
Dies bietet außerdem den Vorteil, dass nur die App selbst darauf zugreifen darf.


\newpage
\section{Modulübersicht}
\include{./subtopicsDesig/	App/Module/Presenter}
\newpage
\section{Klassenübersicht}

\subsection{Gui}
\subsubsection{<<Abstract>>MainActivity} \label{app:klasse:MainActivity}
\textbf{extends} Activity \newline
Die MainActivity bildet die Elternklasse aller weiteren Activities und übernimmt als solche die Navigation durch das Menü. Sie Agiert dabei als OnNavigationItemSelectedListener, implementiert also dessen Methoden.
\newline

\underline{Attribute}
\begin{itemize}
\itemsep0pt
\item \textbf{drawer: NavigationDrawer} \hfill\\ 
\textbf{Sichtbarkeit} private\newline
Die NavigationDrawer Instanz in der das Menü angezeigt wird.

\end{itemize}

\underline{Konstruktoren}\newline
\indent Keine, da der Lebenszyklus dieser Klasse von Android gesteuert wird.\newline

\underline{Methoden}
\begin{itemize}
\itemsep0pt

\item \textbf{onCreate (savedInstanceState: Bundle): void}\hfill\\
\textbf{Sichtbarkeit} public\newline
Überschreibt die onCreate Methode der Superklasse. Ruft die Schablonenmethode \textit{getLayoutRes} auf und initialisiert daraufhin den NavigationDrawer und die Toolbar und setzt sich selbst als OnNavigationItemSelectedListener.

\item \textbf{ <<abstract>> getLayoutRes (): int}\hfill\\
\textbf{Sichtbarkeit} public\newline
Schablonenmethode, welche die Unterklasse das zu ladende Layout bestimmen lässt.

\item \textbf{onNavigationItemSelected(item: MenuItem): boolean}\hfill\\
\textbf{Sichtbarkeit} public\newline
Implementiert das OnNavigationItemSelectedListener Interface und wechselt immer wenn ein Menüeintrag, also Kamera, Videos, Einstellungen, Impressum oder Datenschutz, angeklickt wurden zu der jeweiligen Ansicht.

\end{itemize}
\newpage

\subsubsection{<<abstract>>ContainerActivity}
\textbf{extends} MainActivity \newline
ContainerActivity ist die Elternklasse aller Activities, die eine Toolbar und ein Fragment anzeigen. Das Fragment wird dynamisch geladen, je nach dem welche Ansicht der Nutzer sehen möchte. Hierzu wird ein ``switch-less'' Ansatz verwendet: Die ContainerActivity lädt immer das gleiche Layout und zeigt in dessen Container das Fragment an, welches sie über die Schablonenmethode selectFragment abgefragt hat.
\newline

\underline{Attribute}
\begin{itemize}
\itemsep0pt
\item \textbf{fragment: Fragment} \hfill\\ 
\textbf{Sichtbarkeit} private\newline
Fragment welches aktuell angezeigt werden soll.

\end{itemize}

\underline{Konstruktoren}\newline
\indent Keine, da der Lebenszyklus dieser Klasse von Android gesteuert wird.\newline

\underline{Methoden}
\begin{itemize}
\itemsep0pt

\item \textbf{getLayoutRes (): int}\hfill\\
\textbf{Sichtbarkeit} public\newline
Überschreibt die onResume Methode der Superklasse.

\item \textbf{onCreate (savedInstanceState: Bundle): void}\hfill\\
\textbf{Sichtbarkeit} public\newline
Überschreibt die onResume Methode der Superklasse. Ruft die Methode selectFragment auf und zeigt das erhaltene Fragment an.

\item \textbf{onResume (): void}\hfill\\
\textbf{Sichtbarkeit} public\newline
Überschreibt die onResume Methode der Superklasse. Invalidiert das Fragment.

\label{app:containeractivity:methode:selectfragment} \item \textbf{<<abstract>> selectFragment (): Fragment} \hfill\\
\textbf{Sichtbarkeit} public\newline
Schablonenmethode, welche die Unterklasse das Fragment wählen lässt, das angezeigt werden soll.

\end{itemize}
\newpage

\subsubsection{LogInActivity}
\textbf{extends} MainActivity\newline
LogInActivity behandelt den Anmeldeprozess und ist die erste Activity die gestartet wird. Sie prüft zu aller erst ob der Nutzer bereits vorher seine Anmeldedaten eingegeben hat und reagiert dementsprechent in ihrer onCreate Methode.
\newline

\underline{Attribute}
\begin{itemize}
\itemsep0pt
\item \textbf{logInFragment: LogInFragment} \hfill\\ 
\textbf{Sichtbarkeit} private\newline
Die LogInFragment Instanz
\end{itemize}

\underline{Konstruktoren}\newline
\indent Keine, da der Lebenszyklus dieser Klasse von Android gesteuert wird.\newline

\underline{Methoden}
\begin{itemize}
\itemsep0pt

\item \textbf{Launch (callingActivity: Activity): void}\hfill\\
\textbf{Sichtbarkeit} public
Startet ein neues Intent durch welches eine LogInActivity Instanz erzeugt und gestartet wird.

\item \textbf{getLayoutRes (): int}\hfill\\
\textbf{Sichtbarkeit} public\newline
Überschreibt die getLayoutRes Methode der Superklasse.

\item \textbf{onCreate (savedInstanceState: Bundle): void}\hfill\\
\textbf{Sichtbarkeit} public\newline
Überschreibt die onCreate Methode der Superklasse. Prüft ob der Nutzer angemeldet ist und startet die CameraActivity falls ja. Falls nicht wird das LogInFragment angzeigt.

\end{itemize}
\newpage

\subsubsection{SettingsActivity}
\textbf{extends} ContainerActivity \newline
SettingsActivity zeigt das SettingsFragment an.
\newline

\underline{Attribute}
\begin{itemize}
\itemsep0pt
\item \textbf{settingsFragment: SettingsFragment} \hfill\\ 
\textbf{Sichtbarkeit} private\newline
Die SettingsFragment Instanz

\end{itemize}

\underline{Konstruktoren}\newline
\indent Keine, da der Lebenszyklus dieser Klasse von Android gesteuert wird.\newline

\underline{Methoden}
\begin{itemize}
\itemsep0pt

\item \textbf{Launch (callingActivity: Activity): void}\hfill\\
\textbf{Sichtbarkeit} public\newline
Startet ein neues Intent durch welches eine SettingsActivity Instanz erzeugt und gestartet wird.

\item \textbf{selectFragment (): Fragment}\hfill\\
\textbf{Sichtbarkeit} public\newline
Überschreibt die selectFragment Methode der Superklasse.

\end{itemize}
\newpage

\subsubsection{LegalActivity}
\textbf{extends} ContainerActivity \newline
LegalActivity zeigt das LegalFragment an.
\newline

\underline{Attribute}
\begin{itemize}
\itemsep0pt
\item \textbf{legalFragment: LegalFragment} \hfill\\ 
\textbf{Sichtbarkeit} private\newline
Die LegalFragment Instanz

\end{itemize}

\underline{Konstruktoren}\newline
\indent Keine, da der Lebenszyklus dieser Klasse von Android gesteuert wird.\newline

\underline{Methoden}
\begin{itemize}
\itemsep0pt

\item \textbf{Launch (callingActivity: Activity): void}\hfill\\
\textbf{Sichtbarkeit} public\newline
Startet ein neues Intent durch welches eine LegalActivity Instanz erzeugt und gestartet wird.

\item \textbf{selectFragment (): Fragment}\hfill\\
\textbf{Sichtbarkeit} public\newline
Überschreibt die selectFragment Methode der Superklasse.

\end{itemize}
\newpage

\subsubsection{VideosActivity} \label{app:klasse:VideosActivity}
\textbf{extends} \nameref{app:klasse:ContainerActivity} \newline
Die VideosActivity zeigt das VideosFragment an.
\newline

\underline{Attribute}
\begin{itemize}
\itemsep0pt
\item \textbf{videosFragment: \nameref{app:klasse:VideosFragment}} \hfill\\ 
\textbf{Sichtbarkeit} private\newline
Die VideosFragment Instanz.

\end{itemize}

\underline{Konstruktoren}\newline
\indent Keine, da der Lebenszyklus dieser Klasse von Android gesteuert wird.\newline

\underline{Methoden}
\begin{itemize}
\itemsep0pt

\item \textbf{Launch (callingActivity: Activity): void}\hfill\\
\textbf{Sichtbarkeit} public\newline
Bekommt die Activity übergeben, von der aus der Aufruf ausgeht. Startet ein neues Intent durch welches eine VideosActivity Instanz erzeugt und gestartet wird.

\item \textbf{selectFragment (): Fragment}\hfill\\
\textbf{Sichtbarkeit} public\newline
Überschreibt die \textit{selectFragment} Methode der Superklasse.

\end{itemize}
\newpage

\subsubsection{CameraActivity} \label{app:klasse:CameraActivity}
\textbf{extends} \nameref{app:klasse:MainActivity} \newline
Die CameraActivity zeigt die CameraView an, instanzert eine CompatCameraHandler Instanz sowie eine IRecordCallback Instanz und manipuliert die graphische Nutzeroberfläche abhängig von den Methoden, die auf der IRecordCallback Instanz aufgerufen werden. Nach dem Start blendet die CameraActivity ein Symbol ein, welches die Bereitschaft der App signalisisert. Die CameraActivity stellt einen Observer des CompatCameraHandler dar.
\newline

\underline{Attribute}
\begin{itemize}
\itemsep0pt
\item \textbf{recordCallback: \nameref{app:klasse:IRecordCallback}} \hfill\\ 
\textbf{Sichtbarkeit} private\newline
Implementiert das IRecordCallback Interface. 
Der Aufruf der Mehtode \textit{onRecordStarted} benachrichtigt die CameraActivity Instanz über den Start der Videoaufnahme. Sie blendet das Symbol ein, welches die Aufnahme signalisiert. Das Symbol, welches zuvor die Bereitschaft der App signalisiert hat, wird ausgeblendet.
Der Aufruf der Methode \textit{onRecordStopped} benachrichtigt die CameraActivity Instanz über das Ende der Videoaufnahme. Sie blendet das Symbol ein, welches die Bereitschaft signalisiert. Das Symbol, welches zuvor die Aufnahme der App signalisiert hat, wird ausgeblendet.

\item \textbf{cameraHandler: \nameref{app:klasse:CompatCameraDataHandler}} \hfill\\ 
\textbf{Sichtbarkeit} private\newline
CameraHandler Instanz, die die Vorschaubilder der Kamera eigenständig verarbeitet und die Aufnahme auslößt.

\item \textbf{cameraView: \nameref{app:klasse:CameraView}} \hfill\\ 
\textbf{Sichtbarkeit} private\newline
CameraView Instanz, welche verwendet wird, um die Kameravorschau anzuzeigen.

\end{itemize}

\underline{Konstruktoren}\newline
\indent Keine, da der Lebenszyklus dieser Klasse von Android gesteuert wird.\newline

\underline{Methoden}
\begin{itemize}
\itemsep0pt

\item \textbf{Launch (callingActivity: Activity): void}\hfill\\
\textbf{Sichtbarkeit} public\newline
Startet ein neues Intent durch welches eine CameraActivity Instanz erzeugt und gestartet wird.

\item \textbf{getLayoutRes (): int}\hfill\\
\textbf{Sichtbarkeit} public\newline
Überschreibt die \textit{getLayoutRes} Methode der Superklasse.

\item \textbf{onCreate (savedInstanceState: Bundle): void}\hfill\\
\textbf{Sichtbarkeit} public\newline
Überschreibt die \textit{onCreate} Methode der Superklasse. Lädt die CameraView Instanz und erstellt die IRecorderCallback und CompatCameraHandler Instanzen.

\item \textbf{onResume (): void}\hfill\\
\textbf{Sichtbarkeit} public\newline
Überschreibt die \textit{onResume} Methode der Superklasse. Ruft die Methode \textit{resumeCamera} der CameraHandler Instanz auf.

\item \textbf{onPause (): void}\hfill\\
\textbf{Sichtbarkeit} public\newline
Überschreibt die \textit{onPause} Methode der Superklasse. Ruft die Methode \textit{pauseCamera} der CameraHandler Instanz auf.

\end{itemize}
\newpage

% PUT WEB SERVICE CONTENT HERE
\chapter{Web-Dienst} \label{chap:service}
\chapter{Architektur}
\section{Model View Presenter (MVP)}
Beim Design der Privacy Crash Cam wird schnell deutlich, dass eine strikte Trennung zwischen Datenbank, Applikationslogik und Benutzeroberfläche von Vorteil ist. Neben einer übersichtlichen Unterteilung in einzelne Module unterstützt dieses Vorgehen die Austauschbarkeit der einzelnen Module, deren Wiederverwendbarkeit und Plattformunabhängigkeit. Das Model-View-Presenter-Prinzip realisiert diese Trennung und wird auf auf App und Web interface übertragen.\linebreak\par

\includegraphics[scale=0.5]{./resources/Diagramme/overview_mvp.jpg}

Die Model-Rolle wird vom Web-Dienst erfüllt. Dieser bietet eine REST API, die sowohl vom Web Interface als auch von der App per HTTP-Anfragen angesprochen wird. Falls weitere Software entwickelt wird, mit der auf die Datenbank zugeriffen werden soll, soll diese ebenfalls auf die selbe API zugrifen können.\linebreak
Darüber hanus besitzt die App ein zusätzliches Model-Modul, das den Speicherzugriff regelt.\linebreak\par

Die Apllikationslogik wird durch die Presenter-Ebene umgesetzt. Hierbei besitzen App und Web Interface eigene Presenter-Module, die an die jeweilige Plattform angepasst sind. Der Presenter handhabt Eingaben durch Nutzer oder Sensoren und koordiniert die darauf folgenden Aktionen, wie das Ändern der Ansicht, Aktualisieren der Ansichten und den Zugriff auf die Model Ebene.\linebreak\par 

Die Rolle der View übernimmt schließlich die GUI. In Android wird diese durch entsprechende XML Dokumente implementiert, in Vaadin durch spezielle Java Klassen die von der Vaadin-Internen Klasse ``UI'' erben. Instanzierung, Manipulation und Navigation erfolgen schließlich durch die Presenter-Ebene, die Bindung erfolgt durch Beobachter und ist bereits durch das jewilige Framework festgelegt.


\section{Datenhaltung}

\subsection{Datenbank}

\subsection{temporäre Dateien}
Die Verwaltung der temporären Dateien für die Bearbeitung der Videos übernimmt das \nameref{service:modul:VideoProcessing} Modul. Dies ist notwendig, da die entstehenden und benötigten Daten ausschließlich abhängig von den Arbeitsschritten der \nameref{service:klasse:VideoProcessingChain} ist und daher nicht von anderen Modulen des Web-Dienstes beeinflusst werden soll. Der Web-Dienst stellt dafür den "'tmp"' Ordner bereit.\newline
Der \nameref{service:klasse:EditingContext} erzeugt dann alle für die Bearbeitung notwendigen Dateipfade in der Form:\newline
benutzername\_videoname\_attribut.endung

\subsection{verwendete Resourcen}
Für die Bearbeitung der Videos verwendete Resourcen (z.B. der private Key des \nameref{service:klasse:RSADecryptor} oder der CascadeClassifier des \nameref{service:klasse:ExampleAnalyzer}) werden im "'resources"' Ordner abgelegt.

\section{Modulübersicht} \label{service:modul}

\subsection{Data} \label{service:modul:Data}
Das Data-Modul ist für die Verwaltung der Daten auf der Datenbank da und bietet eine Reihe von Hilfsklassen an, die das behandeln von zusammengehörenden Daten vereinfacht.
\subsection{Manager} \label{service:modul:Manager}
Das Manager-Paket beinhaltet die Presenter-Klassen des Web-Dienstes. Diese dienen dazu die Bearbeitung von Anfragen zu initiieren. So werden Accounts angelegt oder überprüft, hochgeladene Videos zur Bearbeitung weitergegeben, oder Informationen aus der Datenbank abgefragt.

\subsection{ServiceConnector} \label{service:modul:ServiceConnector}
Der ServiceConnector dient der Umsetzung von Rest-Anfragen der App oder des Web-Dienstes. Er selbst besitzt keine Bearbeitungslogik, sondern ent- und verpackt lediglich Anfragen.
\subsection{VideoProcessing} \label{service:modul:VideoProcessing}
Das VideoProcessing-Modul ist ein elementares Modul der Privacy-Crash-Cam. Es ist für die Anonymisierung der, von der \nameref{chap:app} hochgeladenen Videos zuständig.\newline
Das Modul ist in ein Haupt- und ein Unterpaket gegliedert. Das Hauptpaket übernimmt die Verwaltung der Anfragen und die Verteilung der Rechenarbeit auf die vorhandenen Ressourcen. Das Unterpaket \nameref{service:modul:VideoProcessingChain} beinhaltet alle Klassen, die die Bearbeitung des Videos umsetzen.
\subsection{VideoProcessing.Chain} \label{service:modul:VideoProcessingChain}
Das Paket VideoProcessing.Chain beinhaltet alle Klassen, die von der \nameref{service:klasse:VideoProcessingChain} verwendet werden um ein hochgeladenes Video zu bearbeiten.
\section{Klassenübersicht}

\newpage
\subsection{Klasse}
\textbf{extends} Superklasse \newline
\textbf{implements} Interface \newline
Allgemeiner Text bliblablubb.

\subsubsection{Attribute}
\begin{itemize}
\itemsep0pt
\item \textbf{name: Typ} \hfill\\ 
Beschreibung

\item \textbf{name: Typ} \hfill\\ 
Beschreibung
\end{itemize}

\subsubsection{Konstruktoren}
\begin{itemize}
\itemsep0pt
\item \textbf{Klasse()} \hfill\\
Standardkonstruktor
\end{itemize}

\subsubsection{Methoden}
\begin{itemize}
\itemsep0pt
\item \textbf{name (parameter: Typ): Rückgabetyp}\hfill\\
\textbf{Sichtbarkeit} public

Methodenbeschreibung

\item \textbf{name (parameter: Typ): Rückgabetyp}\hfill\\
\textbf{Sichtbarkeit} public

Methodenbeschreibung

\end{itemize}
\newpage


% PUT WEB INTERFACE CONTENT HERE
\chapter{Web-Interface} \label{chap:interface}
\section{Architektur}
Es wurde nach der Model-View-Presenter Architektur entworfen. Wobei das Model hier durch den ServerProxy realisiert wurde, der dann zugriffe auf den Web-Dienst tätigt der die Daten hält. Die View wird durch die Gui und der Presenter durch die ApplicationLogic umgesetzt.

\subsection{Entwurfsmuster}
Im Webservice wurden Entwurfsmuster verwendet um unabhängige Klassen zu erhalten.

\subsubsection{Facade}
Das Entwurfsmuster Facade ist in der ApplicationLogic zu finden. Die beiden DataManager bieten den Views eine Fassade zum zugriff auf den ServerProxy.

\subsubsection{State}
Der Navigator verwendet ein State Entwurfsmuster um die Views anzuzeigen. Die verschiedenen Views entsprechen dabei den States.
\section{Modulübersicht}
Das Modul Webinterface, bietet dem Benutzer eine Schnittstelle zu unserem Web-Dienst, über die er Videos herunterladen und seinen Account verwalten kann. Dieses Modul besteht aus 5 Paketen, Gui, Gui.Navigaiton, ApplicationLogic, ServerConnection und MailService. Es wurde versucht nach der Model-View-Presenter Architektur zu entwerfen. Das Model wird hier durch die Klasse ServerProxy umgesetzt, also ein Stellvertreter zum Web-Dienst, auf dem unsere Daten liegen.
\newpage
\section{Klassenübersicht}

\subsection{Gui}
\subsubsection{UI}\label{UI}
\textbf{extends}  UI \newline
Die UI Klasse bildet das Herzstück des Webinterface. Die init Methode dieser Klasse wird beim Öffnen des Web-Interface aufgerufen. Diese Klasse hat die Aufgabe, alle Komponenten zu initialisieren und bildet dazu die Grundlage für alle graphischen Einheiten.
\newline

\underline{Attribute}
\begin{itemize}
\itemsep0pt

\item \textbf{background: VerticalLayout} \hfill\\ 
Der Background ist die Grundlage der graphischen Oberfläche des Web-Interface. Auf dem Background werden entweder menuArea und contentArea gelegt, oder beim Login die \nameref{LoginView}. Dieses Attribut wird in der init Methode erzeugt und initialisiert.

\item \textbf{menuArea: VerticalLayout} \hfill\\ 
Die menuArea ist die Grundlage für das \nameref{Menu}. Dieses Attribut wird in der init Methode erzeugt und initialisiert.

\item \textbf{contentArea: VerticalLayout} \hfill\\ 
Die contentArea ist die Grundlage für die Ansichten. Dieses Attribut wird in der init Methode erzeugt und initialisiert.

\item \textbf{menu: Menu} \hfill\\ 
Das Menu bildet die Steuereinheit für den Benutzer. Der Navigator wird in der init Methode erzeugt und initialisiert.

\item \textbf{navigator: Navigator} \hfill\\ 
Der Navigator hat die Aufgabe, die verschiedenen Ansichten in die contentArea zu laden. Der Navigator wird in der init Methode erzeugt und initialisiert.

\end{itemize}

\underline{Methoden}
\begin{itemize}
\itemsep0pt
\item \textbf{init (request: VaadinRequest): void}\hfill\\
\textbf{Sichtbarkeit} protected

In dieser Methode wird zuerst die Grundlage für die graphische Oberfläche erzeugt. Dann werden alle graphischen Komponenten, der Navigator und das Menu erzeugt und initialisiert. Am Schluss wird noch die LoginView angezeigt.

\item \textbf{logout (): void}\hfill\\
\textbf{Sichtbarkeit} public

Diese Methode löscht die Account Daten aus dem \nameref{AccountDataManager} und zeigt die LoginView an.

\end{itemize}

\newpage
\newpage
\subsection{LoginView}
\textbf{extends}  VerticalLayout \newline
\textbf{implements} View \newline
Diese Klasse erbt von einem Layout, da sie selbst als graphische Komponente verwendet wird. Die LoginView wird bei jedem Start des Webinterface angezeigt oder nach einem Logout. Nach erfolgreichem Login leitet sie diese Information an eine übergeordnete Komponente weiter.
\newline

\underline{Attribute}
\begin{itemize}
\itemsep0pt
\item \textbf{viewId: String} \hfill\\ 
Die viewId gibt der View eine einzigartige ID über die der Navigator die View identifizieren kann.

\item \textbf{mailField: TextField} \hfill\\ 
Ein einfaches Eingabefeld zur Eingabe der Mail Adresse.

\item \textbf{passwordField: TextField} \hfill\\
Ein einfaches Eingabefeld zur Eingabe des Passworts.

\item \textbf{loginButton: Button} \hfill\\
Dieser Button sendet Mail Adresse und Passwort an den AccountManager zum verifizieren.

\item \textbf{registerButton: Button} \hfill\\
Dieser Button sendet Mail Adresse und Passwort an den Account Manager zum erzeugen eines neuen Accounts.

\end{itemize}

\underline{Konstruktoren}
\begin{itemize}
\itemsep0pt
\item \textbf{LoginView()} \hfill\\
Standardkonstruktor
\end{itemize}

\underline{Methoden}
\begin{itemize}
\itemsep0pt
\item \textbf{enter (viewChangeEvent: ViewChangeListener.ViewChangeEvent): void}\hfill\\
\textbf{Sichtbarkeit} public

Diese Methode wird immer bei eintreten der View aufgerufen.

\item \textbf{update (parameter: void): void}\hfill\\
\textbf{Sichtbarkeit} public

Diese Methode wird verwendet, dass andere Klassen die Möglichkeit haben, diese View zu aktualisieren.


\item \textbf{login (mail: String, password: String) :Boolean} \hfill\\ 
\textbf{Sichtbarkeit} private

Diese Methode wird vom loginButton aufgerufen. Sie sendet mail und password and den AccountManager zur Überprüfung, bei Erfolg wird true zurückgegeben.

\item \textbf{register (mail: String, password: String) :Boolean}\hfill\\
\textbf{Sichtbarkeit} private

Diese Methode wird vom registerButton aufgerufen. Sie sendet mail und password and den AccountManager zur Erstellung eines Accounts, bei Erfolg wird true zurückgegeben.

\end{itemize}

\newpage
\newpage
\subsection{VideoView}\label{VideoView}
\textbf{extends}  VerticalLayout \newline
\textbf{implements} View \newline
Diese Klasse erbt von einem Layout, da sie selbst als graphische Komponente verwendet wird. Diese View dient zum Anzeigen der Videos, die ein Benutzer mit seiner App hochgeladen hat. Zur Anzeige selbst lädt diese Klasse einen \nameref{VideoTable}.
\newline

\underline{Attribute}
\begin{itemize}
\itemsep0pt
\item \textbf{viewId: String} \hfill\\ 
Die viewId gibt der View eine einzigartige ID über die der Navigator die View identifizieren kann.

\end{itemize}

\underline{Konstruktoren}
\begin{itemize}
\itemsep0pt
\item \textbf{VideoView()} \hfill\\
Standardkonstruktor
\end{itemize}

\underline{Methoden}
\begin{itemize}
\itemsep0pt
\item \textbf{enter (viewChangeEvent: ViewChangeListener.ViewChangeEvent) :void}\hfill\\
\textbf{Sichtbarkeit} public

Diese Methode wird immer bei eintreten der View aufgerufen.

\item \textbf{update () :void}\hfill\\
\textbf{Sichtbarkeit} public

Diese Methode wird verwendet zum aktualisieren der View.

\end{itemize}

\newpage
\newpage
\subsubsection{AccountView}\label{AccountView}
\textbf{extends}  VerticalLayout \newline
\textbf{implements} View \newline
Diese Klasse erbt von einem Layout, da sie selbst als graphische Komponente verwendet wird. Diese View dient zur Anzeige der aktuellen Accountdaten und zum Durchführen von Änderungen an diesen.
\newline

\underline{Attribute}
\begin{itemize}
\itemsep0pt
\item \textbf{viewId: String} \hfill\\ 
Die viewId gibt der View eine einzigartige ID, über die der Navigator die View identifizieren kann.

\item \textbf{mailLabel: Label} \hfill\\ 
Dieses Label dient zur Anzeige der derzeit gültigen Mail-Adresse des aktuell eingeloggten \nameref{Account}s.

\item \textbf{passwordChangeField: TextField} \hfill\\ 
In diesem Eingabefeld kann bei Wunsch zur Änderung des Passwortes ein neues Passwort eingegeben werden.

\item \textbf{passwordField: TextField} \hfill\\ 
Für alle gewünschten Änderungen muss hier das aktuell aktive Passwort eingegeben werden.

\item \textbf{mailChangeField: TextField} \hfill\\ 
In diesem Eingabefeld kann bei Wunsch zur Änderung der Mail-Adresse eine neue eingegeben werden

\item \textbf{changeButton: Button} \hfill\\
Durch Drücken dieses Buttons werden die Änderungsdaten an den \nameref{AccountDataManager} zur Bearbeitung geschickt.
\end{itemize}

\underline{Konstruktoren}
\begin{itemize}
\itemsep0pt
\item \textbf{AccountView()} \hfill\\
Standardkonstruktor
\end{itemize}


\underline{Methoden}
\begin{itemize}
\itemsep0pt
\item \textbf{enter (viewChangeEvent: ViewChangeListener.ViewChangeEvent): void}\hfill\\
\textbf{Sichtbarkeit} public

Diese Methode wird immer bei Eintreten der View aufgerufen.

\item \textbf{update (): void}\hfill\\
\textbf{Sichtbarkeit} public

Diese Methode wird zum Aktualisieren der View verwendet.

\end{itemize}

\newpage
\newpage
\subsubsection{ImpressumView}\label{ImpressumView}
\textbf{extends}  VerticalLayout \newline
\textbf{implements} View \newline
Diese Klasse erbt von einem Layout, da sie selbst als graphische Komponente verwendet wird. Diese View hat nur die Aufgabe, das Impressum anzuzeigen. \newline

\underline{Attribute}
\begin{itemize}
\itemsep0pt
\item \textbf{viewId: String} \hfill\\ 
Die viewId gibt der View eine einzigartige ID über die der Navigator die View identifizieren kann.

\item \textbf{impressum: Label)} \hfill\\ 
Dieses Label wird verwendet um das Impressum anzuzeigen.

\end{itemize}

\underline{Konstruktoren}
\begin{itemize}
\itemsep0pt
\item \textbf{ImpressumView()} \hfill\\
Standardkonstruktor
\end{itemize}

\underline{Methoden}
\begin{itemize}
\itemsep0pt
\item \textbf{enter (viewChangeEvent: ViewChangeListener.ViewChangeEvent): void}\hfill\\
\textbf{Sichtbarkeit} public

Diese Methode wird immer bei Eintreten der View aufgerufen.

\item \textbf{update (): void}\hfill\\
\textbf{Sichtbarkeit} public

Diese Methode wird zum Aktualisieren der View verwendet.

\end{itemize}

\newpage
\newpage
\subsection{PrivacyView}\label{PrivacyView}
\textbf{extends}  VerticalLayout \newline
\textbf{implements} View \newline
Diese Klasse erbt von einem Layout, da sie selbst als graphische Komponente verwendet wird. Diese View hat nur die Aufgabe, die Datenschutzinformationen anzuzeigen.
\newline

\underline{Attribute}
\begin{itemize}
\itemsep0pt
\item \textbf{viewId: String} \hfill\\ 
Die viewId gibt der View eine einzigartige ID über die der Navigator die View identifizieren kann.

\item \textbf{privacy: Label} \hfill\\ 
Dieses Label wird verwendet um die Datenschutzinformationen anzuzeigen.

\end{itemize}

\underline{Konstruktoren}
\begin{itemize}
\itemsep0pt
\item \textbf{PrivacyView()} \hfill\\
Standardkonstruktor
\end{itemize}

\underline{Methoden}
\begin{itemize}
\itemsep0pt
\item \textbf{enter (viewChangeEvent: ViewChangeListener.ViewChangeEvent): void}\hfill\\
\textbf{Sichtbarkeit} public

Diese Methode wird immer bei eintreten der View aufgerufen.

\item \textbf{update (): void}\hfill\\
\textbf{Sichtbarkeit} public

Diese Methode wird verwendet, dass andere Klassen die Möglichkeit haben, diese View zu aktualisieren.

\end{itemize}

\newpage
\subsubsection{VideoTable}\label{VideoTable}
\textbf{extends}  Table \newline
Jede Zeile des VideoTables beinhaltet ein \nameref{Video} mit den zugehörigen Buttons zum Downloaden, Löschen und Anzeigen der Infos. \newline

\underline{Attribute}
\begin{itemize}
\itemsep0pt

\item \textbf{downloadButtonList: LinkedList} \hfill\\ 
\textbf{Sichtbarkeit} private

Eine Liste an Buttons. Zu jedem Video wird ein Button erzeugt und an die Liste gehängt.

\item \textbf{infoButtonList: LinkedList} \hfill\\ 
\textbf{Sichtbarkeit} private

Eine Liste an Buttons. Zu jedem Video wird ein Button erzeugt und an die Liste gehängt.

\item \textbf{deleteButtonList: LinkedList} \hfill\\ 
\textbf{Sichtbarkeit} private

Eine Liste an Buttons. Zu jedem Video wird ein Button erzeugt und an die Liste gehängt.

\item \textbf{videos: LinkedList} \hfill\\
\textbf{Sichtbarkeit} private
 
Das Attribut ist eine Liste der Videos. Die Videos werden bei Erzeugen oder Updaten des Tables vom \nameref{VideoDataManager} geholt und dann verarbeitet.
\end{itemize}

\underline{Konstruktoren}
\begin{itemize}
\itemsep0pt

\item \textbf{VideoTable()} \hfill\\ 
\textbf{Sichtbarkeit} public

Im Konstruktor werden die Videos über den VideoDataManager geholt. Zudem werden dann für jedes Video die Buttons vorbereitet.

\end{itemize}


\underline{Methoden}
\begin{itemize}
\itemsep0pt

\item \textbf{prepareVideos (): void}\hfill\\
\textbf{Sichtbarkeit} private

Die Videos werden zur Anzeige vorbereitet.

\item \textbf{prepareButtons (): void}\hfill\\
\textbf{Sichtbarkeit} private

Die Buttons werden zur Anzeige vorbereitet, d.h. es werden Name und Listener gesetzt.

\item \textbf{update (): void}\hfill\\
\textbf{Sichtbarkeit} public

Dies Methode wird verwendet um den Table zu aktualisieren.

\end{itemize}

\newpage
\subsection{Gui.Navigation}
\newpage
\subsection{UI extends UI(com.vaadin.ui.UI)}
Die UI Klasse bildet das Herzstück des Webinterface. Die init[VERLINKEN] Methode dieser Klasse wird beim öffnen des Webinterface aufgerufen. Diese Klasse hat die Aufgabe alle Komponenten zu initialisieren und bildet dazu die Grundlage für alle graphischen Einheiten. Irgendwas zu Servlet sollte hier noch gesagt werden.
\begin{itemize}
\item \subsubsection{Attribute}
\begin{itemize}
\item \textbf{background VerticalLayout(com.vaadin.ui.VerticalLayout)} \hfill\\ 
Der background ist die Grundlage der graphischen Oberfläche des Webinterface. Auf den background werden entweder menuArea und contentArea gelegt, oder beim login die LoginView[VERLINKEN] Dieses Attribut wird in der init[VERLINKEN] Methode erzeugt und initialisiert.

\item \textbf{menuArea VerticalLayout (com.vaadin.ui.VerticalLayout)} \hfill\\ 
Die menuArea ist die Grundlage für das Menü. Dieses Attribut wird in der init[VERLINKEN] Methode erzeugt und initialisiert.

\item \textbf{contentArea VerticalLayout (com.vaadin.ui.UI)} \hfill\\ 
Die contentArea ist die Grundlage für die Ansichten. Dieses Attribut wird in der init[VERLINKEN] Methode erzeugt und initialisiert.

\item \textbf{menu Menu (com.NOCHFESTLEGEN)} \hfill\\ 
Das menu bildet die Steuereinheit für den Benutzer. Der Navigator[VERLINKEN] wird in der init[VERLINKEN] Methode erzeugt und initialisiert.

\item \textbf{navigator Navigator (com.vaadin.navigator.Navigator)} \hfill\\ 
Der Navigator hat die Aufgabe, die verschiedenen Ansichten in die contentArea zu laden. Der Navigator wird in der init[VERLINKEN] Methode erzeugt und initialisiert.

\end{itemize}

\item \subsubsection{Methoden}
\begin{itemize}
\item \textbf{init(VaadinRequest request)} \hfill\\ 
Eingabeparameter: request

Ausgabeparameter:

In dieser Methode wird zuerst die Grundlage für die graphische Oberfläche erzeugt. Dann werden alle graphischen Komponenten, der navigator und das menu erzeugt und initialisiert. Am Schluss wird noch die LoginView angezeigt.

\item \textbf{logout()} \hfill\\ 
Eingabeparameter:

Ausgabeparameter:

Diese Methode löscht die Account Daten aus dem AccountManager und zeigt die LoginView an.

\end{itemize}

\end{itemize}
\newpage
\subsection{DataManagement}
\newpage
\subsection{AccountDataManager}\label{AccountDataManager}
Die Klasse dient zur Accountdatenverwaltung und Kommunikation mit dem \nameref{ServerProxy}. Dazu bereitet sie Daten in beide Richtungen auf.

\underline{Attribute}
\begin{itemize}
\itemsep0pt

\item \textbf{account: \nameref{Account}} \hfill\\ 
Eine Referenz auf den Account, welcher in dieser Sitzung eingeloggt wurde.

\end{itemize}

\underline{Methoden}
\begin{itemize}
\itemsep0pt
\item \textbf{createAccount (mail: String, password: String): String}\hfill\\
\textbf{Sichtbarkeit} public

Eine Methode, die Eingaben der \nameref{LoginView} bekommt und diese dann an den ServerProxy in der jeweiligen Methode weitergibt.

\item \textbf{verifiyAccount (mail: String, password: String): Boolean}\hfill\\
\textbf{Sichtbarkeit} public

Eine Methode die Eingaben der LoginView bekommt und diese dann an den ServerProxy in der jeweiligen Methode weitergibt.

\item \textbf{changeAccount (mail: String, password: String): void}\hfill\\
\textbf{Sichtbarkeit} public

Eine Methode die Eingaben der LoginView bekommt. Anschließend wird das Passwort mit dem derzeitigen Passwort verglichen. Bei Erfolg werden die Änderungen an den ServerProxy übergeben.

\item \textbf{delteAccount (): void}\hfill\\
\textbf{Sichtbarkeit} public

Bei deleteAccount werden die derzeitigen Accountdaten an den ServerProxy in einem delete-Befehl übergeben. Anschließend werden die lokalen Accountdaten gelöscht und die Seite wird neu geladen.

\item \textbf{sendMail (mail: String)}\hfill\\
\textbf{Sichtbarkeit} private

Diese Funktion wird benutzt um Nutzern eine Mail zur Bestätigung nach Erstellen und Löschen eines Accounts zu senden.
\end{itemize}

\newpage
\newpage
\subsubsection{Account}\label{Account}
In dieser Klasse werden die Mail-Adresse und das Passwort eines Benutzers zu einem Account zusammengefasst. \newline

\underline{Attribute}
\begin{itemize}
\itemsep0pt
\item \textbf{mail: String} \hfill\\
\textbf{Sichtbarkeit} private
 
Ein String zum Speichern der Mail-Adresse.

\item \textbf{password: String} \hfill\\ 
\textbf{Sichtbarkeit} private

Ein String zum Speichern des Passwortes.

\end{itemize}

\underline{Konstruktoren}
\begin{itemize}
\itemsep0pt
\item \textbf{Account(String: mail, String: password)} \hfill\\
\textbf{Sichtbarkeit} public

Nimmt die Parameter und setzt die Attribute.
\end{itemize}

\newpage
\newpage
\subsubsection{VideoDataManager}\label{VideoDataManager}
Der VideoDataManager verwaltet die Videodaten und vereinfacht den Zugriff auf den \nameref{ServerProxy} für Klassen, die Videodaten benötigen. \newline

\underline{Attribute}
\begin{itemize}
\itemsep0pt

\item \textbf{videos: LinkedList}\hfill\\
\textbf{Sichtbarkeit} private \newline
\textbf{statisch}

Eine Liste in dem der VideoDataManager die \nameref{Video}s hält die er vom ServerProxy bekommt.
\end{itemize}

\underline{Methoden}
\begin{itemize}
\itemsep0pt


\item \textbf{downloadVideo (videoId: int): void}\hfill\\
\textbf{Sichtbarkeit} public \newline
\textbf{Statisch}

Die Methode fügt die \nameref{Account}-Daten hinzu und ruft danach die Methode zum downloaden am ServerProxy auf.

\item \textbf{deleteVideo (videoId: int): void}\hfill\\
\textbf{Sichtbarkeit} public \newline
\textbf{Statisch}

Die Methode fügt die Account-Daten hinzu und ruft am ServerProxy die Methode zum Löschen eines Videos auf.

\item \textbf{updateVideosAndInfo (): String}\hfill\\
\textbf{Sichtbarkeit} public \newline
\textbf{Statisch}

Die Methode fügt die Account Daten hinzu und ruft am ServerProxy die Methode auf, welcher die Videos zurückgibt. Anschließend wird parseVideos aufgerufen und die Videos als Attribut gespeichert.

\item \textbf{getVideosFromServer (): String}\hfill\\
\textbf{Sichtbarkeit} private \newline
\textbf{Statisch}

Die Methode schickt eine Anfrage an den ServerProxy zum Holen der Videos.

\item \textbf{createVideoList (videos: String): LinkedList}\hfill\\
\textbf{Sichtbarkeit} private \newline
\textbf{Statisch}

Diese Methode parst die Videos in Name und Id. Anschließend erstellt sie eine Liste aus Video Objekten.

\item \textbf{addInfoToVideoList (videos: LinkedList): LinkedList}\hfill\\
\textbf{Sichtbarkeit} private \newline
\textbf{Statisch}

Diese Methode fügt jedem Listeneintrag die Meta-Informationen hinzu.

\item \textbf{getMetaInfFromServer (videoId: int): String}\hfill\\
\textbf{Sichtbarkeit} private \newline
\textbf{Statisch}

Die Methode fügt die Account Daten hinzu und ruft am ServerProxy die Methode auf, die die Metadaten als String zurückgibt.

\end{itemize}
\newpage
\newpage
\subsubsection{Video}\label{Video}
In dieser Klasse werden Name, Id und die Meta-Informationen zu einem Video zusammgefasst. \newline

\underline{Attribute}
\begin{itemize}
\itemsep0pt
\item \textbf{name: String} \hfill\\
\textbf{Sichtbarkeit} private

Ein String zum Speichern des Namens.

\item \textbf{id: int} \hfill\\ 
\textbf{Sichtbarkeit} private

Ein Integer zum Speichern der Id.

\item \textbf{info: String} \hfill\\ 
\textbf{Sichtbarkeit} private

Ein String zum Speichern der Meta-Informationen.

\end{itemize}

\underline{Konstruktoren}
\begin{itemize}
\itemsep0pt
\item \textbf{Account(String: name, int: id, String: info)} \hfill\\
\textbf{Sichtbarkeit} public

Nimmt die Parameter und setzt damit die Attribute.
\end{itemize}

\newpage
\subsection{ServerConnection}
\newpage
\subsection{ServerProxy}\label{ServerProxy}

Diese Klasse dient zur Kommunikation mit dem Webservice. Sie ist die einzige Klasse, welche von ``Außen'' erreichbar ist, bzw. Aktionen mit dem Webservice tätigt.

\underline{Methoden}
\begin{itemize}
\itemsep0pt

\item \textbf{videoDownload (videoId: int, account: \nameref{Account}): Response}\hfill\\
\textbf{Sichtbarkeit} public

Die Methode sendet die Anfrage zum Download eines Videos an den Webservice.

\item \textbf{videoInfo (videoId: int, account: Account): String}\hfill\\
\textbf{Sichtbarkeit} public

Die Methode holt die Metadaten eines Videos vom Webservice.

\item \textbf{videoDelete (videoId: int, account: Account): String}\hfill\\
\textbf{Sichtbarkeit} public

Die Methode sendet die Anfrage zum Löschen eines Videos an den Webservice.

\item \textbf{getVideosByAccount (account: Account): String}\hfill\\
\textbf{Sichtbarkeit} public

Die Methode sendet eine Anfrage an den Webservice, um alle Videos des Accounts zu bekommen.

\item \textbf{authenticteUser (account: Account): String}\hfill\\
\textbf{Sichtbarkeit} public

Die Methode sendet an den Webservice eine Anfrage, die als Antwort liefert, ob die eingegebenen Benutzerdaten gültig sind.

\item \textbf{createAccount (account: Account): String}\hfill\\
\textbf{Sichtbarkeit} public

Die Methode sendet die Anfrage zum Erzeugen eines Accounts~\eqref{Account} an den Webservice.

\item \textbf{changeAccount (accountNew: Account, accountOld: Account): String}\hfill\\
\textbf{Sichtbarkeit} public

Die Methode sendet die Anfrage zum Ändern eines Accounts an den Webservice.

\item \textbf{deleteAccount (account: Account): String}\hfill\\
\textbf{Sichtbarkeit} public

Die Methode sendet die Anfrage zum Löschen eines Accounts an den Webservice.

\item \textbf{verifyAccount (account: Account, checkString: String): String}\hfill\\
\textbf{Sichtbarkeit} public

Die Methode sendet die Anfrage zum Verifizieren eines Accounts an den Webservice.

\end{itemize}

\newpage
\subsection{MailService}
\newpage
\subsection{MailService}\label{MailService}
Diese Klasse dient zum Senden der Verifikationsmail.


\underline{Methoden}
\begin{itemize}
\itemsep0pt
\item \textbf{send (String: to, String: content): void}\hfill\\
\textbf{Sichtbarkeit} public \newline
\textbf{Statisch}

Die Methode konfiguriert den SMTP und erzeugt eine Mail. Anschließend wird die Mail dann über den SMTP gesendet.

\end{itemize}




%PUT ATTACHMENTS HERE
\chapter{Anhang}

\section{Sequenzdiagramme}

\begin{figure}[ht]
	\centering
\includegraphics[width=1\textwidth]{./resources/Diagramme/App/logInSequence.jpg}
\caption{Anmelden in der App}
	\label{fig:AppAuth}
\end{figure}

\begin{figure}[ht]
	\centering
\includegraphics[width=1\textwidth]{./resources/Diagramme/App/recordSequence.jpg}
\caption{Videoaufnahme in der App}
	\label{fig:AppVideo}
\end{figure}

\begin{figure}[ht]
	\centering
\includegraphics[width=1\textwidth]{./resources/Diagramme/App/deleteVideoSequence.jpg}
\caption{Video in der App löschen}
	\label{fig:AppDel}
\end{figure}

\begin{figure}[ht]
	\centering
\includegraphics[width=1\textwidth]{./resources/Diagramme/Webservice/SeqAuthenticate.jpg}
\caption{Authentifizieren auf dem Dienst}
	\label{fig:ServiceAuth}
\end{figure}

\begin{figure}[ht]
	\centering
\includegraphics[width=1\textwidth]{./resources/Diagramme/Webservice/Upload.jpg}
\caption{Video auf den Dienst hochladen}
	\label{fig:ServiceUpload}
\end{figure}

\begin{figure}[ht]
	\centering
\includegraphics[width=1\textwidth]{./resources/Diagramme/Webservice/Processing.jpg}
\caption{Videobearbeitung auf dem Dienst}
	\label{fig:ServiceProcess}
\end{figure}

\begin{figure}[ht]
	\centering
\includegraphics[width=1\textwidth]{./resources/Diagramme/Webservice/SeqVideoDownload.jpg}
\caption{Videodownload vom Dienst}
	\label{fig:ServiceDownl}
\end{figure}

\begin{figure}[ht]
	\centering
\includegraphics[width=1\textwidth]{./resources/Diagramme/Webservice/SeqVideoDeletion.jpg}
\caption{Video auf dem Dienst löschen}
	\label{fig:ServiceDel}
\end{figure}

%end content

\end{document}

