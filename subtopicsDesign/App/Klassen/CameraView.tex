\subsubsection{CameraView} \label{app:klasse:CameraView}
\textbf{extends}CameraView\newline
\textbf{implements} SurfaceHolder.Callback\newline
Die CameraView ist eine SurfaceView, der die Vorschaubilder der Kamera Anzeigt. Da sie das SurfaceHolder.Callback Implementiert wird sie über erstellung, änderung und zerstörung ihres Inhaltes durch Android benachrichtigt.
\newline

\underline{Attribute}
\begin{itemize}
\itemsep0pt
\item \textbf{camera: Camera} \hfill\\ 
\textbf{Sichtbarkeit} private \newline
Kamera Instanz, von der die Vorschaubilder angezeigt werden.

\end{itemize}

\underline{Konstruktoren}
\begin{itemize}
\itemsep0pt
\item \textbf{Camera(context: Context, attrs: AttributeSet)} \hfill\\
\textbf{Sichtbarkeit} public\newline
Konstruktor, der vonAndroid beim Instanzieren der CameraView verwendet wird. Durch verwendung dieses Konstruktors kann die CameraView direkt in die XML-Layout eingebunden werden und muss nicht von der \nameref{app:klasse:CameraActivity} instanziiert werden. 
\end{itemize}

\underline{Methoden}
\begin{itemize}
\itemsep0pt

\item \textbf{setUpCamera(): void}\hfill\\
\textbf{Sichtbarkeit} private\newline
Ruft die Methode \textit{setPreviewDisplay} auf der Camera Instanz und übergibt ihr den per \textit{gerHolder} abgefragten SurfaceHolder. Anschließend wird die Methode \textit{startPreview} auf der Camera Instanz aufgerufen.

\item \textbf{setCamera (camera: Camera): void}\hfill\\
\textbf{Sichtbarkeit} public\newline
Setter, der dem \textit{camera} Feld eine Camera Instanz zuweißt. Falls die CameraView bereits anzegeigt wird, wird die Methode \textit{setUpCamera} aufgerufen.

\item \textbf{surfaceCreated(holder: SurfaceHolder): void}\hfill\\
\textbf{Sichtbarkeit} public\newline
Implementiert die Methode \textit{surfaceCreated} des SurfaceHolder.Callback Interfaces. Falls \textit{camera} nicht \textit{null} ist wird die Methode \textit{setUpCamera} aufgerufen.

\item \textbf{surfaceChanged(holder: SurfaceHolder, format: int, width: int, height: int): void}\hfill\\
\textbf{Sichtbarkeit} public\newline
Implementiert die Methode \textit{surfaceChanged} des SurfaceHolder.Callback Interfaces. Falls \textit{camera} nicht \textit{null} ist wird die Methode \textit{stopPreview} auf der Camera Instanz aufgerufen, das Format auf die erhaltenen Parameter gesetzt und anschließend die Methode \textit{setUpCamera} aufgerufen.

\item \textbf{surfaceDestroyed(holder: SurfaceHolder): void}\hfill\\
\textbf{Sichtbarkeit} public\newline
Implementiert die Methode \textit{surfaceDestroyed} des SurfaceHolder.Callback Interfaces. Aufrufe dieser Methode werden ignoriert.

\end{itemize}