\subsubsection{<<abstract>>ContainerActivity}
\textbf{extends} MainActivity \newline
ContainerActivity ist die Elternklasse aller Activities, die eine Toolbar und ein Fragment anzeigen. Das Fragment wird dynamisch geladen, je nach dem welche Ansicht der Nutzer sehen möchte. Hierzu wird ein ``switch-less'' Ansatz verwendet: Die ContainerActivity lädt immer das gleiche Layout und zeigt in dessen Container das Fragment an, welches sie über die Schablonenmethode selectFragment abgefragt hat.
\newline

\underline{Attribute}
\begin{itemize}
\itemsep0pt
\item \textbf{fragment: Fragment} \hfill\\ 
\textbf{Sichtbarkeit} private\newline
Fragment welches aktuell angezeigt werden soll.

\end{itemize}

\underline{Konstruktoren}\newline
\indent Keine, da der Lebenszyklus dieser Klasse von Android gesteuert wird.\newline

\underline{Methoden}
\begin{itemize}
\itemsep0pt

\item \textbf{getLayoutRes (): int}\hfill\\
\textbf{Sichtbarkeit} public\newline
Überschreibt die onResume Methode der Superklasse.

\item \textbf{onCreate (savedInstanceState: Bundle): void}\hfill\\
\textbf{Sichtbarkeit} public\newline
Überschreibt die onResume Methode der Superklasse. Ruft die Methode selectFragment auf und zeigt das erhaltene Fragment an.

\item \textbf{onResume (): void}\hfill\\
\textbf{Sichtbarkeit} public\newline
Überschreibt die onResume Methode der Superklasse. Invalidiert das Fragment.

\label{app:containeractivity:methode:selectfragment} \item \textbf{<<abstract>> selectFragment (): Fragment} \hfill\\
\textbf{Sichtbarkeit} public\newline
Schablonenmethode, welche die Unterklasse das Fragment wählen lässt, das angezeigt werden soll.

\end{itemize}