\subsubsection{<<Interface>> IServerResponseCallback} \label{app:klasse:IServerResponseCallback}
Das IServerResponseCallback dient zum Beobachten der Serveranfragen. Es wird benachrichtigt, sobald ein Fehler auftritt, ein Fortschritt erzielt wurde oder die Anfrage abgeschlossen ist.
\newline

\underline{Methoden}
\begin{itemize}
\itemsep0pt
\item \textbf{onResponse(result: String)}\hfill\\
\textbf{Sichtbarkeit} public

Aufgerufen, wenn die Netzwerkabfrage eine Antwort geliefert hat. Der Parameter \textit{result} kann auch einen Fehlercode enthalten.

\item \textbf{onProgress(percent: int)}\hfill\\
\textbf{Sichtbarkeit} public

Aufgerufen, wenn der Hintergrundprozess Fortschritt gemacht hat. Bekommt die Prozentzahl des Fortschritts übergeben.

\item \textbf{onError(error: String)}\hfill\\
\textbf{Sichtbarkeit} public

Aufgerufen, wenn beim Ausführen der Anfrage ein Fehler aufgetreten ist. Dieser Fehler bezieht sich nur auf Fehler, die auf dem Gerät selbst aufgetreten sind. Fehler, die der Web-Service meldet, werden hier nicht beachtet.

\end{itemize}