\subsubsection{LegalFragment} \label{app:klasse:LegalFragment}
\textbf{extends} Fragment \newline
Das LegalFragment lädt die einzelnen Layout-Komponenten der Impressumsansicht und reagiert auf Nutzereingaben.
\newline

\underline{Attribute}
\begin{itemize}
\itemsep0pt
\item \textbf{legal: TextView} \hfill\\ 
\textbf{Sichtbarkeit} private \newline
Anklickbarer Text mit Titel ``Impressum''. Zeigt einen Dialog nach Antippen an, in dem eine WebView das Impressum anzeigt.

\item \textbf{privacy: TextView} \hfill\\ 
\textbf{Sichtbarkeit} private \newline
Anklickbarer Text mit Titel ``Datenschutz''. Öffnet die Internetseite mit der Datenschutzerklärung nach Antippen mit dem Standardbrowser.

\item \textbf{licenses: TextView} \hfill\\ 
\textbf{Sichtbarkeit} private \newline
Anklickbarer Text mit Titel ``Lizenzen''. Zeigt einen Dialog nach Antippen an, in dem eine WebView alle Lizenzen aller Drittanbieter-Bibliotheken anzeigt.

\item \textbf{website: TextView} \hfill\\ 
\textbf{Sichtbarkeit} private \newline
Anklickbarer Text mit Titel ``Website''. Öffnet die Internetseite der App nach Antippen mit dem Standardbrowser.

\item \textbf{dialogBuilder: DialogBuilder} \hfill\\ 
\textbf{Sichtbarkeit} private \newline
Wird verwendet um die Dialoge für Impressum und Lizenzen zu erstellen

\item \textbf{content: WebView} \hfill\\ 
\textbf{Sichtbarkeit} private \newline
WebView die durch die Dialoge angezeigt wird.

\item \textbf{loading: ProgressBar} \hfill\\ 
\textbf{Sichtbarkeit} private \newline
Ladebalken, der angezeigt wird, während die Dialoge die WebView Laden.

\end{itemize}

\underline{Konstruktoren}\newline
\indent Keine, da der Lebenszyklus dieser Klasse von Android gesteuert wird.\newline

\underline{Methoden}
\begin{itemize}
\itemsep0pt
\item \textbf{onCreateView (savedInstanceState: Bundle): void}\hfill\\
\textbf{Sichtbarkeit} public\newline
Überschreibt die \textit{onCreateView} Methode der Superklasse. Lädt alle View Instanzen und instanziiert und übergibt jeder View ihren Listener.

\end{itemize}