\subsubsection{LogInActivity} \label{app:klasse:LogInActivity}
\textbf{extends} \nameref{app:klasse:MainActivity}\newline
Die LogInActivity behandelt den Anmeldeprozess und ist die erste Activity die gestartet wird. Sie prüft zuerst, ob der Nutzer bereits vorher seine Anmeldedaten eingegeben hat und reagiert dementsprechend in ihrer \textit{onCreate} Methode.
\newline

\underline{Attribute}
\begin{itemize}
\itemsep0pt
\item \textbf{logInFragment: \nameref{app:klasse:LogInFragment}} \hfill\\ 
\textbf{Sichtbarkeit} private\newline
Die LogInFragment Instanz, die angezeigt wird.

\item \textbf{logInHelper: \nameref{app:klasse:LogInHelper}} \hfill\\ 
\textbf{Sichtbarkeit} private\newline
Die LogInHelper Instanz, mit der Accountdaten gespeichert werden und mit der überprüft wird, ob ein Nutzer angemeldet ist.

\end{itemize}

\underline{Konstruktoren}\newline
\indent Keine, da der Lebenszyklus dieser Klasse von Android gesteuert wird.\newline

\underline{Methoden}
\begin{itemize}
\itemsep0pt

\item \textbf{Launch (callingActivity: Activity): void}\hfill\\
\textbf{Sichtbarkeit} public \newline
\textbf{statisch} 

Startet ein neues Intent durch welches eine LogInActivity Instanz erzeugt und gestartet wird.

\item \textbf{getLayoutRes (): int}\hfill\\
\textbf{Sichtbarkeit} public\newline
Überschreibt die getLayoutRes() Methode der Superklasse.

\item \textbf{onCreate (savedInstanceState: Bundle): void}\hfill\\
\textbf{Sichtbarkeit} public\newline
Überschreibt die onCreate() Methode der Superklasse. Prüft ob der Nutzer angemeldet ist und startet die \nameref{app:klasse:CameraActivity} falls ja. Falls nicht wird das LogInFragment angzeigt.

\end{itemize}