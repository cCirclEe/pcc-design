\subsubsection{LogInFragment} \label{app:klasse:LogInFragment}
\textbf{extends} Fragment \newline
Das LogInFragment lädt die einzelnen Layout-Komponenten der Anmeldeansicht und reagiert auf Nutzereingaben. Es steuert den Kontrollfluss des Anmeldevorgangs und aktualisiert die Bedienoberfläche dementsprechend.
\newline
Siehe auch: \nameref{fig:AppAuth}\newline

\underline{Attribute}
\begin{itemize}
\itemsep0pt
\item \textbf{mail: EditText} \hfill\\ 
\textbf{Sichtbarkeit} private \newline
Eingabefeld für die Email-Adresse des Nutzers.

\item \textbf{password: EditText} \hfill\\ 
\textbf{Sichtbarkeit} private \newline
Eingabefeld für das Passwort des Nutzers.

\item \textbf{logIn: Button} \hfill\\ 
\textbf{Sichtbarkeit} private \newline
Schaltfläche um den Anmeldevorgang zu starten.

\item \textbf{register: Button} \hfill\\ 
\textbf{Sichtbarkeit} private \newline
Schaltfläche um zur Registrierung zu gelangen.

\item \textbf{status: ProgressBar} \hfill\\ 
\textbf{Sichtbarkeit} private \newline
Ladebalken, der während der Überprüfung der Accountdaten angezeigt wird.

\item \textbf{memoryManager: \nameref{app:klasse:MemoryManager}} \hfill\\ 
\textbf{Sichtbarkeit} private \newline
MemoryManager Instanz um die Accountdaten zu speichern.

\item \textbf{server: \nameref{app:klasse:ServerProxy}} \hfill\\ 
\textbf{Sichtbarkeit} private \newline
ServerProxy Instanz um Anfragen an den Server zu stellen.

\item \textbf{serverResponse: \nameref{app:klasse:IServerResponseCallback}} \hfill\\ \textbf{Sichtbarkeit} private \newline
Implementiert das IServerResponseCallback Interface. Beim Aufruf der onResponse() Methode wird die Antwort des Servers ausgelesen. Bestätigt der Server die eingegebenen Accountdaten ruft das LogInFragment die Launch() Methode der \nameref{app:klasse:CameraActivity} auf. Andernfalls wird dem Nutzer durch einen Toast eine Fehlermeldung angezeigt. Beim Aufruf der onError() Methode wird dem Nutzer die entsprechende Fehlermeldung ebenfalls per Toast angezeigt. Die onProgress() Methode wird ignoriert.

\end{itemize}

\underline{Konstruktoren}\newline
\indent Keine, da der Lebenszyklus dieser Klasse von Android gesteuert wird.\newline

\underline{Methoden}
\begin{itemize}
\itemsep0pt
\item \textbf{onCreateView (savedInstanceState: Bundle): void}\hfill\\
\textbf{Sichtbarkeit} public\newline
Überschreibt die onCreateView() Methode der Superklasse. Lädt alle View Instanzen und instanziiert und übergibt jeder View ihren Listener. Holt sich eine MemoryManager und eine ServerProxy Instanz und instanziiert den IServerResponseCallback.

\item \textbf{performLogIn (view: View): void}\hfill\\
\textbf{Sichtbarkeit} public\newline
Aufgerufen, wenn der Nutzer den Anmeldevorgang startet. Bekommt die View die angeklickt wurde übergeben. Ließt die Texte der TextView Instanzen aus, validiert die Eingaben und ruft die authenticateAccount() Methode auf der ServerProxy Instanz auf. 

\item \textbf{redirectToRegistry (view: View): void}\hfill\\
\textbf{Sichtbarkeit} public\newline
Aufgerufen, wenn der Nutzer auf die Registrieren-Schaltfläche tippt. Bekommt die View die angeklickt wurde übergeben. Öffnet die Internetseite zur Registrierung mit dem Standardbrowser des Smartphones.

\end{itemize}