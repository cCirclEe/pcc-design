\subsubsection{<<Abstract>>MainActivity} \label{app:klasse:MainActivity}
\textbf{extends} Activity \newline
Die MainActivity bildet die Elternklasse aller weiteren Activities und übernimmt als solche die Navigation durch das Menü. Sie Agiert dabei als OnNavigationItemSelectedListener, implementiert also dessen Methoden.
\newline

\underline{Attribute}
\begin{itemize}
\itemsep0pt
\item \textbf{drawer: NavigationDrawer} \hfill\\ 
\textbf{Sichtbarkeit} private\newline
Die NavigationDrawer Instanz in der das Menü angezeigt wird.

\end{itemize}

\underline{Konstruktoren}\newline
\indent Keine, da der Lebenszyklus dieser Klasse von Android gesteuert wird.\newline

\underline{Methoden}
\begin{itemize}
\itemsep0pt

\item \textbf{onCreate (savedInstanceState: Bundle): void}\hfill\\
\textbf{Sichtbarkeit} public\newline
Überschreibt die onCreate Methode der Superklasse. Ruft die Schablonenmethode \textit{getLayoutRes} auf und initialisiert daraufhin den NavigationDrawer und die Toolbar und setzt sich selbst als OnNavigationItemSelectedListener.

\item \textbf{ <<abstract>> getLayoutRes (): int}\hfill\\
\textbf{Sichtbarkeit} public\newline
Schablonenmethode, welche die Unterklasse das zu ladende Layout bestimmen lässt.

\item \textbf{onNavigationItemSelected(item: MenuItem): boolean}\hfill\\
\textbf{Sichtbarkeit} public\newline
Implementiert das OnNavigationItemSelectedListener Interface und wechselt immer wenn ein Menüeintrag, also Kamera, Videos, Einstellungen, Impressum oder Datenschutz, angeklickt wurden zu der jeweiligen Ansicht.

\end{itemize}