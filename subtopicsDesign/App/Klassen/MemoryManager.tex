\subsubsection{TMemoryManager} \label{app:klasse:MemoryManager}
Die MemoryManager-Klasse verwaltet die Daten auf der App. Er bietet die Funktionalität bestimmte Daten anzufordern, zu sichern oder zu löschen. Die Klasse ist ein Singelton, somit existiert nur eine Instanz mit der man die folgenden Methoden aufrufen kann.\newline

\underline{Attribute}
\begin{itemize}
\itemsep0pt
\item \textbf{instance: MemoryManager} \hfill\\ 
\textbf{Sichtbarkeit} private

Repräsentiert die Singleton Instanz.

\item \textbf{context: Context} \hfill\\ 
\textbf{Sichtbarkeit} private

Kontext der App. Beinhaltet die benötigten Speicherdirektorien um Daten in den richtigen Speicherplätzen abzulegen bzw. wiederzufinden.
\end{itemize}

\underline{Konstruktoren}
\begin{itemize}
\itemsep0pt
\item \textbf{MemoryManager(context: Context)} \hfill\\
\textbf{Sichtbarkeit} private

Privater Konstruktor der Klasse, der nur aufgerufen wird wenn das instance-Attribut der Klasse noch nicht gesetzt wurde.
\end{itemize}

\underline{Methoden}
\begin{itemize}
\itemsep0pt

\item \textbf{GetInstance(context: Context): MemoryManager}\hfill\\
\textbf{Sichtbarkeit} public

Statische Methode um sicherzustellen, dass es nur eine Instanz der Klasse gibt. Wenn Instanz noch nicht erzeugt wurde, ruft die Methode den Konstruktor auf. Das instance-Attribut wird dann auf die erzeugte Instanz gesetzt und zurückgegeben. Wenn instance bereits erzeugt wurde, wird dieses direkt zurückgegeben.

\item \textbf{getTempVideoFile(): File}\hfill\\
\textbf{Sichtbarkeit} public

Gibt die temporäre Videodatei in Form eines Files zurück.

\item \textbf{getTempSymmetricKeyFile(): File}\hfill\\
\textbf{Sichtbarkeit} public

Gibt die temporäre symmetrische Schlüsseldatei in Form eines Files zurück.

\item \textbf{deleteTempData()}\hfill\\
\textbf{Sichtbarkeit} public

Löscht alle temporären Daten.

\item \textbf{saveAccountData(account: Account)}\hfill\\
\textbf{Sichtbarkeit} public

Sichert die Accountdaten die per Parameterübergabe mitgegeben werden.

\item \textbf{getAccountData(): Account}\hfill\\
\textbf{Sichtbarkeit} public

Gibt eine Instanz der Klasse Account, in der sich die Accountdaten befinden.

\item \textbf{deleteAccountData()}\hfill\\
\textbf{Sichtbarkeit} public

Löscht die Accountdaten des Benutzers.

\item \textbf{saveSettings(settings: Settings)}\hfill\\
\textbf{Sichtbarkeit} public

Sichert die Einstellungen die über Parameterübergabe mitgegeben werden.

\item \textbf{getSettings(): Settings}\hfill\\
\textbf{Sichtbarkeit} public

Gibt eine Instanz der Klasse Settings zurück, welche die Einstellungen repräsentieren.

\item \textbf{saveEncryptedSymmetricKey(videoName: String, key: String): File}\hfill\\
\textbf{Sichtbarkeit} public

Sichert den verschlüsselten symmetrischen Schlüssel eines Videos und gibt als Rückgabe den Schlüssel als File zurück. 

\item \textbf{saveEncryptedVideo(videoName: String, byte[]): File}\hfill\\
\textbf{Sichtbarkeit} public

Sichert das verschlüsselte Video und gibt als Rückgabe das Video als File zurück.

\item \textbf{saveEncryptedMetadata(videoName: String, metadata: Metadata): File}\hfill\\
\textbf{Sichtbarkeit} public

Sichert die verschlüsselten Metadaten eines Videos und gibt als Rückgabe die Daten als File zurück.

\item \textbf{saveReadableMetadata(videoName: String, metadata: Metadata): File}\hfill\\
\textbf{Sichtbarkeit} public

Sichert die lesbaren Metadaten eines Videos gibt als Rückgabe die Daten als File zurück.

\item \textbf{deleteEncryptedSymmetricKey(videoName: String)}\hfill\\
\textbf{Sichtbarkeit} public

Löscht den verschlüsselten symmetrischen Schlüssel.

\item \textbf{deleteEncryptedVideo(videoName: String)}\hfill\\
\textbf{Sichtbarkeit} public

Löscht das verschlüsselte Video.

\item \textbf{deleteEncryptedMetadata(videoName: String)}\hfill\\
\textbf{Sichtbarkeit} public

Löscht die verschlüsselten Metadaten.

\item \textbf{deleteReadableMetadata(videoName: String)}\hfill\\
\textbf{Sichtbarkeit} public

Löscht die lesbaren Metadaten.

\item \textbf{getAllVideos(): ArrayList<Video>}\hfill\\
\textbf{Sichtbarkeit} public

Gibt alle Videos in Form einer ArrayList zurück.

\item \textbf{getEncryptedSymmetricKey(videoName: String): File}\hfill\\
\textbf{Sichtbarkeit} public

Gibt den verschlüsselten symmetrischen Schlüssel zu einem Video als File zurück.

\item \textbf{getEncryptedVideo(name: String): File}\hfill\\
\textbf{Sichtbarkeit} public

Gibt die verschlüsselte Videodatei als File zurück.

\item \textbf{getEncryptedMetadata(videoName: String): File}\hfill\\
\textbf{Sichtbarkeit} public

Gibt die verschlüsselten Metadaten zu einem Video als File zurück.

\item \textbf{getReadableMetadata(videoName: String): Metadata}\hfill\\
\textbf{Sichtbarkeit} public

Gibt die lesbaren Metadaten zu einem Video als File zurück.
\end{itemize}