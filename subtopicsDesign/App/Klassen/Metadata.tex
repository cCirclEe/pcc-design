\subsubsection{Metadata} \label{app:klasse:Metadata}
Datencontainer für Videometadaten. \newline

\underline{Attribute}
\begin{itemize}
\itemsep0pt
\item \textbf{date: long} \hfill\\ 
\textbf{Sichtbarkeit} private

Aufnahmedatum des Videos

\item \textbf{triggerType: String} \hfill\\ 
\textbf{Sichtbarkeit} private

Auslöser der Videoaufnahme

\item \textbf{force: float[ ]} \hfill\\ 
\textbf{Sichtbarkeit} private

G-Sensorwerte zum Auslösezeitpunkt
\end{itemize}

\underline{Konstruktoren}
\begin{itemize}
\itemsep0pt
\item \textbf{Metadata(date: long, triggerType: String, force: float[ ])} \hfill\\
\textbf{Sichtbarkeit} public

Weist die Parameter den Attributen zu.

\item \textbf{Metadata(json: String)} \hfill\\
\textbf{Sichtbarkeit} public

Liest die Parameter aus dem json String und weist sie den Attributen zu.
\end{itemize}

\underline{Methoden}
\begin{itemize}
\itemsep0pt
\item \textbf{getAsJSON(): String}\hfill\\
\textbf{Sichtbarkeit} public

Verpackt die Attribute als json String und gibt diesen zurück.

\end{itemize}