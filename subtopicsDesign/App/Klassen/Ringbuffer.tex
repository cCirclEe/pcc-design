\subsubsection{Ringbuffer<E>} \label{app:klasse:Ringbuffer}
Der Ringbuffer ist eine nach Ankunftszeit geordnete Datenstruktur, die wenn sie Daten annimmt immer das älteste Element löscht, falls ihre Kapazität überschritten wird. \newline

\underline{Attribute}
\begin{itemize}
\itemsep0pt
\item \textbf{length: int} \hfill\\ 
\textbf{Sichtbarkeit} private

Kapazität des Puffers

\item \textbf{data: Queue<E>} \hfill\\ 
\textbf{Sichtbarkeit} private

Interne Datenstruktur des Puffers
\end{itemize}

\underline{Konstruktoren}
\begin{itemize}
\itemsep0pt
\item \textbf{Ringbuffer(length: int)} \hfill\\
\textbf{Sichtbarkeit} public

Erstellt einen neuen leeren Ringpuffer mit der angegebenen Kapazität.
\end{itemize}

\underline{Methoden}
\begin{itemize}
\itemsep0pt
\item \textbf{offer (element: E): boolean}\hfill\\
\textbf{Sichtbarkeit} public

Fügt das neue Element am Ende der Datenstruktur hinzu. Falls dadurch die Kapazität überschritten wird, wird am Anfang der Datenstruktur ein Element entfernt. Gibt zurück, ob das Einfügen erfolgreich war.

\item \textbf{getData (): Queue<E>}\hfill\\
\textbf{Sichtbarkeit} public

Gibt die Daten des Ringpuffers zurück.

\end{itemize}