\subsubsection{Settings} \label{app:klasse:Settings}
Datencontainer für Appeinstellungen. \newline

\underline{Attribute}
\begin{itemize}
\itemsep0pt
\item \textbf{fps: int} \hfill\\ 
\textbf{Sichtbarkeit} private

Bildwiederholrate

\item \textbf{bufferSizeSec: Typ} \hfill\\ 
\textbf{Sichtbarkeit} private

Länge des aufgenommenen Videos bei Persistierung.

\item \textbf{quality: int} \hfill\\ 
\textbf{Sichtbarkeit} private

Bildqualität der Aufnahme

\end{itemize}

\underline{Konstruktoren}
\begin{itemize}
\itemsep0pt
\item \textbf{Settings(fps: int, bufferSizeSec: Typ, quality: int)} \hfill\\
\textbf{Sichtbarkeit} public

Weist die Parameter den Attributen zu.

\item \textbf{Settings(json: String)} \hfill\\
\textbf{Sichtbarkeit} public

Liest die Parameter aus dem json String und weist sie den Attributen zu.
\end{itemize}

\underline{Methoden}
\begin{itemize}
\itemsep0pt
\item \textbf{setResolution (resulution: Vector2D): Rückgabetyp}\hfill\\
\textbf{Sichtbarkeit} public

Ändert die Bildqualität

\item \textbf{getAsJSON (): String}\hfill\\
\textbf{Sichtbarkeit} public

Verpackt die Attribute als json String und gibt diesen zurück.

\end{itemize}