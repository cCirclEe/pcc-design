\subsubsection{SettingsFragment} \label{app:klasse:SettingsFragment}
\textbf{extends} Fragment \newline
Das SettingsFragment lädt die einzelnen Layout-Komponenten der Einstellungenansicht, reagiert auf Nutzereingaben und veranlasst das Speichern gemachter Änderungen. Jede antippbare Layout-Komponente bekommt einen anonymen OnClickListener der die mit der jeweiligen Komponente verknüpfte Aktion ausführt.
\newline

\underline{Attribute}
\begin{itemize}
\itemsep0pt
\item \textbf{settings: \nameref{app:klasse:Settings} } \hfill\\ 
\textbf{Sichtbarkeit} private \newline
Die aktuellen Einstellungen, gekapselt in einer Settings Instanz.

\item \textbf{qualityLow: TextView} \hfill\\ 
\textbf{Sichtbarkeit} private \newline
Anklickbarer Text um niedrige Auflösung festzulegen. Antippen ändert den in der Settings Instanz gespeicherten Wert \textit{quality}, färbt diese TextView grün und \textit{qualityMedium} und \textit{qualityHigh} grau.

\item \textbf{qualityMedium: TextView} \hfill\\ 
\textbf{Sichtbarkeit} private \newline
Anklickbarer Text um mittlere Auflösung festzulegen. Antippen ändert den in der Settings Instanz gespeicherten Wert \textit{quality}, färbt diese TextView grün und \textit{qualityLow} und \textit{qualityHigh} grau.

\item \textbf{qualityHigh: TextView} \hfill\\ 
\textbf{Sichtbarkeit} private \newline
Anklickbarer Text um hohe Auflösung festzulegen. Antippen ändert den in der Settings Instanz gespeicherten Wert \textit{quality}, färbt diese TextView grün und \textit{qualityLow} und \textit{qualityMedium} grau.

\item \textbf{fps: TextView} \hfill\\ 
\textbf{Sichtbarkeit} private \newline
Text zum Anzeigen der aktuell eingestellten Bilder pro Sekunde.

\item \textbf{fpsBar: SeekBar} \hfill\\ 
\textbf{Sichtbarkeit} private \newline
Balken zum Tippen und Ziehen um die Einstellung der  Bilder pro Sekunde zu ändern. Antippen ändert den in der Settings Instanz gespeicherten Wert \textit{fps} und aktualisiert \textit{fps}.

\item \textbf{bufferSize: TextView} \hfill\\ 
\textbf{Sichtbarkeit} private \newline
Text zum Anzeigen der aktuell eingestellten Buffergröße.

\item \textbf{incBuffer: Button} \hfill\\ 
\textbf{Sichtbarkeit} private \newline
Schaltfläche um die Buffergröße zu erhöhen. Antippen erhöht die in der Settings Instanz gespeicherten Wert \textit{bufferSize} und aktualisiert die TextView \textit{bufferSize}.

\item \textbf{decBuffer: Button} \hfill\\ 
\textbf{Sichtbarkeit} private \newline
Schaltfläche um die Buffergröße zu verringern. Antippen verringert die in der Settings Instanzgespeicherten Wert \textit{bufferSize} und aktualisiert die TextView \textit{bufferSize}.

\item \textbf{logOut: Button} \hfill\\ 
\textbf{Sichtbarkeit} private \newline
Schaltfläche um AccountDaten zu löschen und alle Appansichten außer die LogInActivity unzugänglich zu machen. Antippen ruft die Methode \textit{deleteAccountData} des MemoryManagers und die Methode \textit{Launch} der \nameref{app:klasse:LogInActivity} auf.

\item \textbf{memoryManager: MemoryManager} \hfill\\ 
\textbf{Sichtbarkeit} private \newline
MemoryManager Instanz um gespeicherte Einstellungen abzufragen und neue Einstellugen zu speichern.

\end{itemize}

\underline{Konstruktoren}\newline
\indent Keine, da der Lebenszyklus dieser Klasse von Android gesteuert wird.\newline

\underline{Methoden}
\begin{itemize}
\itemsep0pt
\item \textbf{onCreateView (savedInstanceState: Bundle): void}\hfill\\
\textbf{Sichtbarkeit} public\newline
Überschreibt die \textit{onCreateView} Methode der Superklasse. Lädt alle View Instanzen, instanziiert und übergibt jeder View ihren Listener, holt sich eine MemoryManager Instanz und ruft die \textit{getSettings} Methode des MemoryManagers auf. Wenn sie das Settings Objekt von diesem Methodenaufruf erhalten hat weißt sie es der \textit{settings} Variable zu und aktualisiert alle View-Komponenten, so dass sie die in \textit{settings} gespeicherten Daten wiederspiegeln.

\item \textbf{onPause (): void}\hfill\\
\textbf{Sichtbarkeit} public\newline
Überschreibt die \textit{onCreateView} Methode der Superklasse. Ruft die \textit{saveSettings} Methode des MemoryManagers auf und übergibt das Settings Objekt.

\end{itemize}