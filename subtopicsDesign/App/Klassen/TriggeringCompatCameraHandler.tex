\subsubsection{TriggeringCompatCameraHandler} \label{app:klasse:TriggeringCompatCameraHandler}
\textbf{extends} \nameref{app:klasse:CompatCameraHandler} \newline
\textbf{implements} SensorEventListener, View.OnClickListener \newline
Der TriggeringCompatCameraHandler ist ein CompatCameraHandler, der seine Videoaufnahme nach Nutzereingabe, oder nachdem die gemessenen Beschleunigungswerte des G-Sensors einen Maximalwert überschreiten, auslößt.
\newline

\underline{Attribute}
\begin{itemize}
\itemsep0pt
\item \textbf{maxGForce: float} \hfill\\ 
\textbf{Sichtbarkeit} private \newline
Maximalwert für die gemessene Beschleunigung. Überschreiten die Messwerte des Sensors diesen Wert wird die Videoaufnahme gestartet.

\item \textbf{accelerometerSensor: Sensor} \hfill\\ 
\textbf{Sichtbarkeit} private \newline
Beschleunigungssensor, der überwacht werden soll. CompatCameraDataHandler ist ein Observer dieses Sensors.

\item \textbf{sensormanager: SensorManager} \hfill\\ 
\textbf{Sichtbarkeit} private \newline
SensorManager, der verwendet wird, um den \textit{accelerometerSensor} zuzuweisen und den TriggeringCompatCameraHandler als SensorEventListener zu registrieren.
\end{itemize}

\underline{Konstruktoren}
\begin{itemize}
\itemsep0pt
\item \textbf{CompatCameraHandler(context: Context, preview: CameraView, recordCallback: \nameref{app:klasse:IRecordCallback})} \hfill\\
\textbf{Sichtbarkeit} public\newline
Konstruktor, der \textit{super} aufruft und der Elternklasse die erhaltenen Parameter übergibt.
\end{itemize}

\underline{Methoden}
\begin{itemize}
\itemsep0pt

\item \textbf{onSensorChanged (event: SensorEvent): void}\hfill\\
\textbf{Sichtbarkeit} public\newline
Implementiert die onSensorChanged() Methode des SensorEventListener Interfaces. Überprüft das \textit{values} Feld des erhaltenen SensorEvents und vergleicht die Einträge dieses Arrays mit \textit{maxGForce}. Falls die gemessenen Werte \textit{maxGForce} überschreiten wird eine Metadata Instanz erstellt und die Methoden setMetadata(..) und schedulePersisting() aufgerufen, die in der Elternklasse implementiert sind.

\item \textbf{onClick (view: View): void}\hfill\\
\textbf{Sichtbarkeit} public\newline
Implementiert die onClick() Methode des View.OnClickListener Interfaces. Beim Aufruf wird eine Metadata Instanz erstellt und die Methoden setMetadata(..) und schedulePersisting() aufgerufen, die in der Elternklasse implementiert sind.

\end{itemize}