\newpage
\subsubsection{AccountView}\label{AccountView}
\textbf{extends}  VerticalLayout \newline
\textbf{implements} View \newline
Diese Klasse erbt von einem Layout, da sie selbst als graphische Komponente verwendet wird. Diese View dient zur Anzeige der aktuellen Accountdaten und zum Durchführen von Änderungen an diesen.
\newline

\underline{Attribute}
\begin{itemize}
\itemsep0pt
\item \textbf{viewId: String} \hfill\\
\textbf{Sichtbarkeit} private
 
Die viewId gibt der View eine einzigartige ID, über die der Navigator die View identifizieren kann.

\item \textbf{mailLabel: Label} \hfill\\ 
\textbf{Sichtbarkeit} private

Dieses Label dient zur Anzeige der derzeit gültigen Mail-Adresse des aktuell eingeloggten \nameref{Account}s.

\item \textbf{passwordChangeField: TextField} \hfill\\ 
\textbf{Sichtbarkeit} private

In diesem Eingabefeld kann bei Wunsch zur Änderung des Passwortes ein neues Passwort eingegeben werden.

\item \textbf{passwordField: TextField} \hfill\\ 
\textbf{Sichtbarkeit} private

Für alle gewünschten Änderungen muss hier das aktuell aktive Passwort eingegeben werden.

\item \textbf{mailChangeField: TextField} \hfill\\ 
\textbf{Sichtbarkeit} private

In diesem Eingabefeld kann bei Wunsch zur Änderung der Mail-Adresse eine neue eingegeben werden

\item \textbf{changeButton: Button} \hfill\\
\textbf{Sichtbarkeit} private

Durch Drücken dieses Buttons werden die Änderungsdaten an den \nameref{AccountDataManager} zur Bearbeitung geschickt.
\end{itemize}

\underline{Konstruktoren}
\begin{itemize}
\itemsep0pt
\item \textbf{AccountView()} \hfill\\
\textbf{Sichtbarkeit} public

Standardkonstruktor
\end{itemize}

\underline{Methoden}
\begin{itemize}
\itemsep0pt
\item \textbf{enter (viewChangeEvent: ViewChangeListener.ViewChangeEvent): void}\hfill\\
\textbf{Sichtbarkeit} public

Diese Methode wird immer bei Eintreten der View aufgerufen.

\item \textbf{update (): void}\hfill\\
\textbf{Sichtbarkeit} public

Diese Methode wird zum Aktualisieren der View verwendet.

\end{itemize}
