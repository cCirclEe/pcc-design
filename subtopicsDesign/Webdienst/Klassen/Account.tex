\subsubsection{Account} \label{service:klasse:Account}
Die Klasse Account repräsentiert den Account eines Benutzers. \newline

\underline{Attribute}
\begin{itemize}
\itemsep0pt
\item \textbf{email: String} \hfill\\ 
\textbf{Sichtbarkeit} private 

E-Mail-Adresse des Accounts.

\item \textbf{passwordHash: String} \hfill\\ 
\textbf{Sichtbarkeit} private 

PasswordHash des Accounts.

\item \textbf{id: int} \hfill\\ 
\textbf{Sichtbarkeit} private  

Die ``id'' ist identisch mit der ``id'' in der Tabelle User der Datenbank.
\end{itemize}

\underline{Konstruktoren}
\begin{itemize}
\itemsep0pt
\item \textbf{Account(json: String)} \hfill\\
\textbf{Sichtbarkeit} public

Der Konstruktor nimmt ein JSON-Objekt entgegen, wertet die Informationen aus und setzt die Attribute. Die Information ``password'' wird mit der Methode ``hashPassword(password: String)'' gehasht und dann gesetzt.
\end{itemize}

\underline{Methoden}
\begin{itemize}
\itemsep0pt
\item \textbf{hashPassword(password: String): String}\hfill\\
\textbf{Sichtbarkeit} private

Die Methode nimmt das im Klartext gegebene Passwort und hasht dieses.
\end{itemize}