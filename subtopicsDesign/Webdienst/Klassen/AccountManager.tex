\subsubsection{AccountManager} \label{service:klasse:AccountManager}
Verbindet zwischen dem \nameref{service:modul:Data}-Modul und dem \nameref{service:modul:Server}-Modul. Er bearbeitet die Anfragen des \nameref{service:klasse:ServerProxy}s und gibt die gefragten Ergebnisse zurück. \newline

\underline{Attribute}
\begin{itemize}
\itemsep0pt
\item \textbf{account: \nameref{service:klasse:Account}} \hfill\\ 
\textbf{Sichtbarkeit} public

Account-Instanz, welche zur Bearbeitung der verschiedenen Anfragen benötigt wird.
\end{itemize}

\underline{Konstruktoren}
\begin{itemize}
\itemsep0pt
\item \textbf{AccountManager(account: Account)} \hfill\\
\textbf{Sichtbarkeit} public

Erstellt eine Instanz eines AccountManagers mit dem dazugehörigen Account. Dieser wird vom ServerProxy über die Parameter übergeben und dann gesetzt.
\end{itemize}

\underline{Methoden}
\begin{itemize}
\itemsep0pt
\item \textbf{setMail(newMail: String): String}\hfill\\
\textbf{Sichtbarkeit} public

Bearbeitet eine Anfrage vom ServerProxy zur Ersetzung einer Mail eines Accounts. Die übergebene Mail wird dann im \nameref{service:klasse:DatabaseManager} mit der Methode setMail() gesetzt.

\item \textbf{setPassword(newPasswordHash: String): String}\hfill\\
\textbf{Sichtbarkeit} public

Bearbeitet eine Anfrage vom ServerProxy zur Ersetzung eines Passwort eines Accounts. Das übergebene Passwort wird dann im DatabaseManager mit der Methode setPassword() gesetzt.

\item \textbf{getAccountId(): int}\hfill\\
\textbf{Sichtbarkeit} public

Bearbeitet eine Anfrage vom ServerProxy zur Bestimmung der AccountId zu einer Mail-Adresse. Die Anfrage wird zum DatabaseManager weitergeleitet und somit die Methode getAccountId() aufgerufen. Von dort aus wird sie bis zum ServerProxy zurückgegeben.  

\item \textbf{registerAccount(uuid :String): String}\hfill\\
\textbf{Sichtbarkeit} public

Bearbeitet eine Anfrage vom ServerProxy zur Accountregistrierung. Die Die Übergabe ``uuid'' stellt eine eindeutige Id des Accounts dar, die zur Accountverifizierung dient. Dabei wird beim DatabaseManager die Methode register() aufgerufen und die Accountdaten im zugehörigen Accountattribut gesetzt. 

\item \textbf{deleteAccount(vm :VideoManager): String}\hfill\\
\textbf{Sichtbarkeit} public

Bearbeitet die Löschung eines Accounts. Zunächst werden alle Videos des Accounts ermittelt und es wird mit dem Übergabeparameter ``vM'' auf den \nameref{service:klasse:VideoManager} zugegriffen, der alle Videos und Metadaten löscht. Nun wird im DatabaseManager ``deleteAccount()'' aufgerufen und der Account wird in der Datenbank gelöscht.

\item \textbf{authenticate(): boolean}\hfill\\
\textbf{Sichtbarkeit} public

Bearbeitet eine Anfrage vom ServerProxy zur Authentifizierung des Accounts. Die Methode authenticate() vom DatabaseManager wird hierbei aufgerufen.
 
\item \textbf{verifyAccount(accountData :String, uuid: String) :String}\hfill\\
\textbf{Sichtbarkeit} public

Bearbeitet eine Anfrage zur Verifizierung der E-Mail eines Accounts. Die Übergabe ``uuid'' stellt eine eindeutige Id des Accounts dar. Die Methode gibt einen JSON-String mit dem Ergebnis zurück. 
 
\item \textbf{isVerfied(): boolean}\hfill\\
\textbf{Sichtbarkeit} public

Bearbeitet eine Anfrage vom ServerProxy zum Verifizierungsstatus eines Accounts. Die Anfrage wird zum DatabaseManager weitergeleitet und die Methode isVerified() aufgerufen. Von dort wird der boolean bis zum ServerProxy zurückgegeben.
\end{itemize}