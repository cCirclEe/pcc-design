\subsubsection{Anonymizer} \label{service:klasse:Anonymizer}
\textbf{extends} \nameref{service:klasse:AAnonymizer} \newline
Der Anonymizer bietet eine mögliche konkrete Implementierung eines Video-Anonymizers. Er arbeitet indem er das Video frameweise analysiert, um dann die für die Anonymisierung relevanten Bildbereiche mit einem Filter unkenntlich zu machen. Genutzt werden hierfür OpenCV Schnittstellen.\newline

\underline{Attribute}
\begin{itemize}
\itemsep0pt
\item \textbf{analyzer: IAnalyzer} \hfill\\ 
\textbf{Sichtbarkeit} private

Analysierungsalgorithmus zum Erkennen der für die Anonymisierung relevanten Bildbereiche.

\item \textbf{filter: IFilter} \hfill\\ 
\textbf{Sichtbarkeit} private

Bildfilter zum Anonymisieren von Bildbereichen.
\end{itemize}

\underline{Konstruktoren}
\begin{itemize}
\itemsep0pt
\item \textbf{Anonymizer()} \hfill\\
\textbf{Sichtbarkeit} public

Standardkonstruktor
\end{itemize}

\underline{Methoden}
\begin{itemize}
\itemsep0pt
\item \textbf{anonymize(input: File, 
output: File): boolean}\hfill\\
\textbf{Sichtbarkeit} public

Implementiert die abstrakte Methode anonymize() von AAnonymizer. Liest das Video frameweise ein und über gibt den Frame zunächst an den analyzer. Die dort erkannten relevanten Bildbereiche werden dann an den filter gegeben. Das Video wird dann wieder Frame für Frame zusammengeführt und gespeichert. Gibt zurück, ob das anonymisieren erfolgreich war.

\end{itemize}