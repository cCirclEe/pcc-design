\subsubsection{DatabaseManager} \label{service:klasse:DatabaseManager}
Die Klasse DatabaseManager bietet eine Schnittstelle für alle Datenbankanfragen, die vom AccountManager und VideoManager benötigt wird. \hfill\\

\underline{Attribute}
\begin{itemize}
\itemsep0pt
\item \textbf{account: Account} \hfill\\ 
\textbf{Sichtbarkeit} private

Benutzeraccount des aktiven Nutzers
\end{itemize}

\underline{Konstruktoren}
\begin{itemize}
\itemsep0pt
\item \textbf{DatabaseManager(account: Account)} \hfill\\
Setzt den aktiven Nutzer.
\end{itemize}

\underline{Methoden}
\begin{itemize}
\itemsep0pt
\item \textbf{saveProcessedVideoAndMeta(videoName: String, metaName: String): boolean}\hfill\\
\textbf{Sichtbarkeit} public

Bekommt den Videofile-Namen, sowie die Metadaten als übergeben und schreibt einen Datenbank-Eintrag in die Video-Tabelle der Datenbank. Die ``id'' wird generiert, die ``user\_id'' ist das Attribut ``id'' des Account-Objektes, der ``video\_name'' ist der Übergabeparamter ``videoName'' und ``meta\_name'' ist der Übergabeparamter ``metaName''.

\item \textbf{getVideoInfo(videoId: int): VideoInfo}\hfill\\
\textbf{Sichtbarkeit} public

Bbekommt die videoId übergeben und gibt Videoinformationen als VideoInfo-Objekt zurück. Die Daten werden mithilfe der videoId und den Accountdaten des Nutzers aus der Datenbank geladen.

\item \textbf{getVideoInfoList(): ArrayList<VideoInfo>}\hfill\\
\textbf{Sichtbarkeit} public

Gibt alle Videos eines Benutzers in Form einer Liste von ``VideoInfo''-Objekten zurück. Diese wird durch eine Datenbankabfrage mithilfe der Accountdaten ermittelt.

\item \textbf{deleteVideoAndMeta(videoId: int): boolean}\hfill\\
\textbf{Sichtbarkeit} public

Löscht ein Video eines Benutzers. Das Löschen geschieht durch eine Datenbankabfrage mithilfe der videoId und den Accountdaten des aktiven Benutzers. Die Methode gibt zurück, ob das Löschen erfolgreich war.

\item \textbf{getMetadata(videoId: int): Metadata}\hfill\\
\textbf{Sichtbarkeit} public

Ermittelt die Metadaten eines Videos durch eine Datenbankabfrage. Dafür wird die videoId und die Benutzerdaten genutzt.

\item \textbf{setMail(newMail: String): boolean}\hfill\\
\textbf{Sichtbarkeit} public

Die Methode ändert die E-Mail-Adresse des Benutzers. Die E-Mail-Adresse wird als ``newMail'' übergeben und mithilfe des Accounts kann auf das Element in der Datenbank zugegriffen werden. Mit dem entsprechenden Datenbankbefehl wird die neue E-Mail gesetzt. Es wird zurückgegeben, ob das aktualisieren erfolgreich war, oder nicht.

\item \textbf{setPassword(newPassword: String): boolean}\hfill\\
\textbf{Sichtbarkeit} public

Die Methode ändert das Passwort des Benutzers. Der neue Passwort-Hash wird als Parameter ``newPasswordHash'' übergeben und mithilfe des Accounts kann durch ein Datenbankbefehl der neue Passwort-Hash gesetzt werden. Ein ``boolean'' wird zurückgegeben, je nachdem, ob die Operation erfolgreich war oder nicht.

\item \textbf{authenticate(): boolean}\hfill\\
\textbf{Sichtbarkeit} public

Die Methode authentifiziert den Benutzer. Durch das den Account stehen alle benötigten Informationen zur Verfügung. Durch eine Datenbankabfrage wird überprüft, ob E-Mail und Passwort-Hash übereinstimmen. Es wird zurückgegeben, ob das aktualisieren erfolgreich war, oder nicht.

\item \textbf{deleteAccount(): boolean}\hfill\\
\textbf{Sichtbarkeit} public

Die Methode löscht einen Account. Durch den Account stehen alle Informationen zur Verfügung. Zunächst werden alle Videos und Metadaten des Nutzers gelöscht. Danach werden die Video-Datenbankeinträge in der Tabelle ``Video'' und dann der Account in der Tabelle ``User'' gelöscht.

\item \textbf{getAccountId() :int}\hfill\\
\textbf{Sichtbarkeit} public

Die Methode gibt die ``id'' des Accounts zurück. Diese ermittelt sie durch eine Datenbankabfrage mit der Adresse des Accounts. Falls der Account nicht existiert, wird ``-1'' zurückgegeben.

\item \textbf{register(uuid :String) :boolean}\hfill\\
\textbf{Sichtbarkeit} public

Die Methode legt einen neuen Benutzer an. Die Die Übergabe ``uuid'' ist ein eindeutiger Wert, der zur Accountverifizierung dient. Die zur Erstellung des Account benötigten Informationen liegen in dem Attribut account vor. Daraus werden E-Mail und Passwort genommen und mit einer Datenbankabfrage wird in der Tabelle ``User'' ein neuer Eintrag hinzugefügt. 

\item \textbf{verifyAccount(accountData :String, uuid: String): String}\hfill\\
\textbf{Sichtbarkeit} public

Bearbeitet eine Anfrage zur Verifizierung der E-Mail eines Accounts. Die Übergabe ``uuid'' stellt eine Eindeutige id des Accounts dar. Die Methode gibt einen JSON-String mit dem Ergebnis zurück.

\item \textbf{isVerfied(): boolean}\hfill\\
\textbf{Sichtbarkeit} public

Die Methode überprüft, ob ein Nutzer verifiziert ist.

\end{itemize}