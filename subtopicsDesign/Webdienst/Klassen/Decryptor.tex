\subsubsection{Decryptor} \label{service:klasse:Decryptor}
\textbf{implements} \nameref{service:klasse:IStage} \newline
Der Decryptor ist dafür da, das hybride Verschlüsselungsverfahren der App~\eqref{chap:app} rückgängig zu machen.\newline

\underline{Attribute}
\begin{itemize}
\itemsep0pt
\item \textbf{fileDecryptor} \hfill \\
\textbf{Sichtbarkeit} private

Decryptor für symmetrisch verschlüsselte Dateien.

\item \textbf{keyDecryptor} \hfill \\
\textbf{Sichtbarkeit} private

Decryptor für asymmetrisch verschlüsselte SecretKeys.

\end{itemize}

\underline{Konstruktoren}
\begin{itemize}
\itemsep0pt
\item \textbf{Decryptor()} \hfill\\
\textbf{Sichtbarkeit} public

Standardkonstruktor
\end{itemize}

\underline{Methoden}
\begin{itemize}
\itemsep0pt
\item \textbf{execute (): boolean}\hfill\\
\textbf{Sichtbarkeit} public

Überschreibt die Methode execute() von IStage. Ruft decrypt(..) auf.

\item \textbf{decrypt (encVid: File, encKey: File, 
encMeta: File, decVid: File): boolean}\hfill\\
\textbf{Sichtbarkeit} public

Erzeugt zunächst den symmetrischen und asymmetrischen Decryptor. Gibt daraufhin den verschlüsselten SecretKey an den keyDecryptor und entschlüsselt mit diesem dann im fileDecryptor die Video- und die Metadata-Datei. Gibt zurück, ob das Entschlüsseln erfolgreich war.

\end{itemize}