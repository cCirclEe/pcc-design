\subsubsection{Persistor} \label{service:klasse:Persistor}
\textbf{implements} \nameref{service:klasse:IStage} \newline
Der Persistor fügt die Metadaten in die Video-Datei ein und legt das Video langfristig auf dem Video ab. Dazu sogt er dafür, dass das Video in die Database aufgenommen wird.\newline

\underline{Konstruktoren}
\begin{itemize}
\itemsep0pt
\item \textbf{Persistor()} \hfill\\

Standardkonstruktor
\end{itemize}

\underline{Methoden}
\begin{itemize}
\itemsep0pt
\item \textbf{execute (context: Context): boolean}\hfill\\
\textbf{Sichtbarkeit} public

Implementiert die Methode execute(..) von IStage. Ruft die Methode anonymize(..) auf.

\item \textbf{persist (video: File, metadata: Metadata): boolean}\hfill\\
\textbf{Sichtbarkeit} public

Lädt das Video und fügt alle Metadaten in die Metadaten des Videos ab. Daraufhin wird das Video mithilfe des \nameref{service:klasse:DatabaseManager}s in den permanenten Speicher des Services abgelegt und der Datenbank hinzugefügt.

\end{itemize}