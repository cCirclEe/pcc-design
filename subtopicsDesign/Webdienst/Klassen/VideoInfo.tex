\subsubsection{VideoInfo} \label{service:klasse:VideoInfo}
Die Klasse VideoInfo beinhaltet die benötigten Informationen zum Video.\newline

\underline{Attribute}
\begin{itemize}
\itemsep0pt
\item \textbf{videoId: int} \hfill\\ 
\textbf{Sichtbarkeit} private \hfill\\  

Die ``videoId'' ist die ``videoId'' des zugehörigen Datenbankeintrags.

\item \textbf{videoName: String} \hfill\\ 
\textbf{Sichtbarkeit} private \hfill\\ 

Der ``videoName'' ist der Name des Videos im zugehörigen Datenbankeintrag.

\end{itemize}

\underline{Konstruktoren}
\begin{itemize}
\itemsep0pt
\item \textbf{Video(videoId: int, videoName: String)} \hfill\\

Weist die Parameterübergabe den zugehörigen Klassenattributen zu.

\end{itemize}

\underline{Methoden}
\begin{itemize}
\itemsep0pt
\item \textbf{getAsJson(): String}\hfill\\
\textbf{Sichtbarkeit} public

Die Methode gibt ein JSON-String zurück, der aus den Informationen aller Klassenattributen zusammengesetzt wird.

\end{itemize}