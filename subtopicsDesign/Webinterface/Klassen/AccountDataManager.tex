\newpage
\subsection{AccountDataManager}\label{AccountDataManager}
Die Klasse dient zur Accountdatenverwaltung und Kommunikation mit dem ServerProxy~\eqref{ServerProxy}. Dazu bereitet sie Daten in beide Richtungen auf.

\underline{Attribute}
\begin{itemize}
\itemsep0pt

\item \textbf{account: Account~\eqref{Account}} \hfill\\ 
Eine Referenz auf den Account der in dieser Session eingeloggt wurde.

\end{itemize}

\underline{Methoden}
\begin{itemize}
\itemsep0pt
\item \textbf{createAccount (mail: String, password: String): String}\hfill\\
\textbf{Sichtbarkeit} public

Eine Methode die Eingaben der LoginView~\eqref{LoginView} bekommt und diese dann an den ServerProxy~\eqref{ServerProxy} in der jeweiligen Methode weitergibt.

\item \textbf{verifiyAccount (mail: String, password: String): Boolean}\hfill\\
\textbf{Sichtbarkeit} public

Eine Methode die Eingaben der LoginView~\eqref{LoginView} bekommt und diese dann an den ServerProxy~\eqref{ServerProxy} in der jeweiligen Methode weitergibt.

\item \textbf{changeAccount (mail: String, password: String): void}\hfill\\
\textbf{Sichtbarkeit} public

Eine Methode die Eingaben der LoginView~\eqref{LoginView} bekommt. Anschließend wird das Passwort mit dem derzeitigen Passwort verglichen, bei Erfolg werden die Änderungen an den ServerProxy~\eqref{ServerProxy} übergeben.

\item \textbf{delteAccount (): void}\hfill\\
\textbf{Sichtbarkeit} public

Bei deleteAccount werden die derzeitigen Accountdaten an den ServerProxy~\eqref{ServerProxy} in einem delete Befehl übergeben. Anschließend werden die lokalen Accountdaten gelöscht und die Seite wird neu geladen. 
(Der Account~\eqref{Account} wurde damit aus der Datenbank gelöscht)

\item \textbf{sendMail (mail: String)}\hfill\\
\textbf{Sichtbarkeit} private

Diese Funktion wird benutzt um Nutzern eine Mail zur Bestätigung nach erstellen und löschen eines Accounts~\eqref{Account} zu senden.
\end{itemize}