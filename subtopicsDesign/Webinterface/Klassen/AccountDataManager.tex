\subsubsection{AccountDataManager}\label{AccountDataManager}
Die Klasse dient zur Accountdatenverwaltung und Kommunikation mit dem \nameref{ServerProxy}. Dazu bereitet sie Daten in beide Richtungen auf. \newline

\underline{Attribute}
\begin{itemize}
\itemsep0pt

\item \textbf{account: \nameref{Account}} \hfill\\ 
\textbf{Sichtbarkeit} private \newline
\textbf{statisch}

Eine Referenz auf den Account, der in dieser Sitzung eingeloggt wurde.

\end{itemize}

\underline{Methoden}
\begin{itemize}
\itemsep0pt
\item \textbf{createAccount (mail: String, password: String): String}\hfill\\
\textbf{Sichtbarkeit} public \newline
\textbf{Statisch}

Eine Methode, die Eingaben der \nameref{LoginView} bekommt und diese dann an den ServerProxy in der jeweiligen Methode weitergibt.

\item \textbf{startVerification (mail: String, password: String): void}\hfill\\
\textbf{Sichtbarkeit} private \newline
\textbf{Statisch}

Diese Methode wird nach dem Registrieren aufgerufen. Sie erzeugt eine UUID und sendet diese einmal in einem Link per Mail an den Benutzer und dann noch an den ServerProxy.

\item \textbf{authenticteAccount (account: Account): boolean}\hfill\\
\textbf{Sichtbarkeit} public \newline
\textbf{Statisch}

Diese Methode wird beim Login benutzt. Sie gibt an ob Passwort und Mail-Adresse korrekt sind.

\item \textbf{checkVerification (account: Account): boolean}\hfill\\
\textbf{Sichtbarkeit} public \newline
\textbf{Statisch}

Diese Methode wird beim Login benutzt. Sie gibt an ob ein Account verifiziert ist.

\item \textbf{changeAccount (mail: String, password: String): void}\hfill\\
\textbf{Sichtbarkeit} public \newline
\textbf{Statisch}

Eine Methode die Eingaben von der \nameref{AccountView} bekommt. Anschließend wird das Passwort mit dem derzeitigen Passwort verglichen. Bei Erfolg werden die Änderungen an den ServerProxy übergeben.

\item \textbf{deleteAccount (): void}\hfill\\
\textbf{Sichtbarkeit} public \newline
\textbf{Statisch}

Bei deleteAccount werden die derzeitigen Account-Daten an den ServerProxy in einem delete-Befehl übergeben. Anschließend werden die lokalen Account Daten gelöscht und die Seite neu geladen.

\item \textbf{sendMail (mail: String): void}\hfill\\
\textbf{Sichtbarkeit} private \newline
\textbf{Statisch}

Diese Funktion wird benutzt um Nutzern eine Mail zur Bestätigung nach Erstellen und Löschen eines Accounts zu senden.
\end{itemize}
