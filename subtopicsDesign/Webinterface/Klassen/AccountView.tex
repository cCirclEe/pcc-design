\newpage
\subsection{AccountView extends VerticalLayout implements View}
Diese Klasse erbt von einem Layout, da sie selbst als graphische Komponente verwendet wird. Diese View dient zur Anzeige der aktuellen Accountdaten und zum Durchführen von Änderungen an diesen.

\begin{itemize}
\item \subsubsection{Attribute}
\begin{itemize}
\item \textbf{viewId String} \hfill\\ 
Die viewId gibt der View eine einzigartige ID über die der Navigator die View identifizieren kann.

\item \textbf{mailLabel Label (com.vaadin.ui.Label)} \hfill\\ 
Dieses Label dient zur anzeige der derzeit gültigen Mail Adresse des aktuell eingeloggten Accounts.

\item \textbf{passwordChangeField TextField (com.vaadin.ui.TextField)} \hfill\\ 
In diesem Eingabe Feld kann bei Wunsch zur Änderung des Passwortes ein neues Passwort eingegeben werden.

\item \textbf{passwordField TextField (com.vaadin.ui.TextField)} \hfill\\ 
Für alle gewünschten Änderungen muss hier das aktuell aktive Passwort Eingegeben werden.

\item \textbf{mailChangeField TextField (com.vaadin.ui.TextField)} \hfill\\ 
In diesem Eingabe Feld kann bei Wunsch zur Änderung der Mail Adresse eine neue eingegeben werden

\item \textbf{changeButton Button (com.vaadin.ui.Button)} \hfill\\
Durch drücken dieses Buttons werden die Änderungsdaten an den AccountManager zur Bearbeitung geschickt.

\end{itemize}

\item \subsubsection{Methoden}
\begin{itemize}
\item \textbf{enter(ViewChangeListener.ViewChangeEvent viewChangeEvent)}\hfill\\
Type: public

Eingabeparameter: viewChangeEvent

Ausgabeparameter:

Diese Methode wird immer bei eintreten der View aufgerufen.



\item \textbf{update()} \hfill\\ 
Type: public

Eingabeparameter:

Ausgabeparameter:

Diese Methode wird verwendet, dass andere Klassen die Möglichkeit haben, diese View zu aktualisieren.
\end{itemize}

\end{itemize}