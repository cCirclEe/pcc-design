\newpage
\subsection{AccountView}\label{AccountView}
\textbf{extends}  VerticalLayout \newline
\textbf{implements} View \newline

Diese Klasse erbt von einem Layout, da sie selbst als graphische Komponente verwendet wird. Diese View dient zur Anzeige der aktuellen Accountdaten und zum Durchführen von Änderungen an diesen.
\newline

\underline{Attribute}
\begin{itemize}
\itemsep0pt
\item \textbf{viewId: String} \hfill\\ 
Die viewId gibt der View eine einzigartige ID über die der Navigator die View identifizieren kann.

\item \textbf{mailLabel: Label} \hfill\\ 
Dieses Label dient zur anzeige der derzeit gültigen Mail Adresse des aktuell eingeloggten Accounts~\eqref{Account}.

\item \textbf{passwordChangeField: TextField} \hfill\\ 
In diesem Eingabe Feld kann bei Wunsch zur Änderung des Passwortes ein neues Passwort eingegeben werden.

\item \textbf{passwordField: TextField} \hfill\\ 
Für alle gewünschten Änderungen muss hier das aktuell aktive Passwort Eingegeben werden.

\item \textbf{mailChangeField: TextField} \hfill\\ 
In diesem Eingabe Feld kann bei Wunsch zur Änderung der Mail Adresse eine neue eingegeben werden

\item \textbf{changeButton: Button} \hfill\\
Durch drücken dieses Buttons werden die Änderungsdaten an den AccountDataManager~\eqref{AccountDataManager} zur Bearbeitung geschickt.
\end{itemize}

\underline{Konstruktoren}
\begin{itemize}
\itemsep0pt
\item \textbf{AccountView()} \hfill\\
Standardkonstruktor
\end{itemize}


\underline{Methoden}
\begin{itemize}
\itemsep0pt
\item \textbf{enter (viewChangeEvent: ViewChangeListener.ViewChangeEvent): void}\hfill\\
\textbf{Sichtbarkeit} public

Diese Methode wird immer bei eintreten der View aufgerufen.

\item \textbf{update (): void}\hfill\\
\textbf{Sichtbarkeit} public

Diese Methode wird verwendet, dass andere Klassen die Möglichkeit haben, diese View zu aktualisieren.

\end{itemize}