\newpage
\subsection{LoginView}\label{LoginView}
\textbf{extends}  VerticalLayout \newline
\textbf{implements} View \newline
Diese Klasse erbt von einem Layout, da sie selbst als graphische Komponente verwendet wird. Die \nameref{LoginView} wird bei jedem Start des Webinterface angezeigt oder nach einem Logout. Nach erfolgreichem Login leitet sie diese Information an eine übergeordnete Komponente weiter.
\newline

\underline{Attribute}
\begin{itemize}
\itemsep0pt
\item \textbf{viewId: String} \hfill\\ 
Die viewId gibt der View eine einzigartige ID über die der Navigator die View identifizieren kann.

\item \textbf{mailField: TextField} \hfill\\ 
Ein einfaches Eingabefeld zur Eingabe der Mail Adresse.

\item \textbf{passwordField: TextField} \hfill\\
Ein einfaches Eingabefeld zur Eingabe des Passworts.

\item \textbf{loginButton: Button} \hfill\\
Dieser Button sendet Mail Adresse und Passwort an den AccountManager zum verifizieren.

\item \textbf{registerButton: Button} \hfill\\
Dieser Button sendet Mail Adresse und Passwort an den \nameref{AccountDataManager} zum erzeugen eines neuen \nameref{Account}s.

\end{itemize}

\underline{Konstruktoren}
\begin{itemize}
\itemsep0pt
\item \textbf{LoginView(UI ui)} \hfill\\
Durch die Referenz auf die UI wird nach erfolgreichem Login das \nameref{Menu} und die \nameref{VideoView} geladen.
\end{itemize}

\underline{Methoden}
\begin{itemize}
\itemsep0pt
\item \textbf{enter (viewChangeEvent: ViewChangeListener.ViewChangeEvent): void}\hfill\\
\textbf{Sichtbarkeit} public

Diese Methode wird immer bei eintreten der View aufgerufen.

\item \textbf{update (): void}\hfill\\
\textbf{Sichtbarkeit} public

Diese Methode wird verwendet, dass andere Klassen die Möglichkeit haben, diese View zu aktualisieren.


\item \textbf{login (mail: String, password: String): Boolean} \hfill\\ 
\textbf{Sichtbarkeit} private

Diese Methode wird vom loginButton aufgerufen. Sie sendet mail und password and den AccountDataManager zur Überprüfung, bei Erfolg wird true zurückgegeben.

\item \textbf{register (mail: String, password: String): Boolean}\hfill\\
\textbf{Sichtbarkeit} private

Diese Methode wird vom registerButton aufgerufen. Sie sendet mail und password and den AccountDataManager zur Erstellung eines Accounts, bei Erfolg wird true zurückgegeben.

\end{itemize}
