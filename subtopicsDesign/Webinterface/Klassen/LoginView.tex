\newpage
\subsubsection{LoginView}\label{LoginView}
\textbf{extends}  VerticalLayout \newline
\textbf{implements} View \newline
Diese Klasse erbt von einem Layout, da sie selbst als graphische Komponente verwendet wird. Die \nameref{LoginView} wird bei jedem Start des Web-Interface oder nach einem Logout angezeigt. Nach erfolgreichem Login leitet sie diese Information an eine übergeordnete Komponente weiter.
\newline

\underline{Attribute}
\begin{itemize}
\itemsep0pt
\item \textbf{viewId: String} \hfill\\ 
\textbf{Sichtbarkeit} private

Die viewId gibt der View eine einzigartige ID über die der Navigator die View identifizieren kann.

\item \textbf{mailField: TextField} \hfill\\ 
\textbf{Sichtbarkeit} private

Ein einfaches Eingabefeld zur Eingabe der Mail-Adresse.

\item \textbf{passwordField: TextField} \hfill\\
\textbf{Sichtbarkeit} private

Ein einfaches Eingabefeld zur Eingabe des Passwortes.

\item \textbf{loginButton: Button} \hfill\\
Dieser Button sendet Mail-Adresse und Passwort an den AccountDataManager zum verifizieren.

\item \textbf{registerButton: Button} \hfill\\
\textbf{Sichtbarkeit} private

Dieser Button sendet Mail-Adresse und Passwort an den \nameref{AccountDataManager} zum erzeugen eines neuen \nameref{Account}s.

\end{itemize}

\underline{Konstruktoren}
\begin{itemize}
\itemsep0pt
\item \textbf{LoginView(ui: \nameref{MyUI})} \hfill\\
\textbf{Sichtbarkeit} public

Durch die Referenz auf die UI wird nach erfolgreichem Login das \nameref{Menu} und die \nameref{VideoView} geladen.
\end{itemize}

\underline{Methoden}
\begin{itemize}
\itemsep0pt
\item \textbf{enter (viewChangeEvent: ViewChangeListener.ViewChangeEvent): void}\hfill\\
\textbf{Sichtbarkeit} public

Diese Methode wird immer bei Eintreten der View aufgerufen.

\item \textbf{update (): void}\hfill\\
\textbf{Sichtbarkeit} public

Diese Methode wird zum Aktualisieren der View verwendet.

\item \textbf{login (mail: String, password: String): Boolean} \hfill\\ 
\textbf{Sichtbarkeit} private

Diese Methode wird vom loginButton aufgerufen. Sie sendet zur Überprüfung Mail und Passwort and den AccountDataManager. Bei Erfolg wird true zurückgegeben.

\item \textbf{register (mail: String, password: String): Boolean}\hfill\\
\textbf{Sichtbarkeit} private

Diese Methode wird vom registerButton aufgerufen. Sie sendet Mail und Passwort and den AccountDataManager zur Erstellung eines Accounts. Bei Erfolg wird true zurückgegeben.

\end{itemize}
