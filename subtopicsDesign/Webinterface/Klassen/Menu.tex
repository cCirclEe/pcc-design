\newpage
\subsection{Menu}\label{Menu}
\textbf{extends}  HorizontalLayout \newline
Das \nameref{Menu} stellt die Buttons bereit, die benötigt werden, um den Benutzer zwischen den verschiedenen Ansichten wechseln zu lassen. Dazu kommt noch der Logout Button, über den man zurück zum Login gelangt.
\newline

\underline{Attribute}
\begin{itemize}
\itemsep0pt

\item \textbf{menuItemLayout: Layout} \hfill\\ 
In dieses Layout werden neu erstellte Menu-Items hinzugefügt.

\item \textbf{userMenu: MenuBar} \hfill\\ 
Eine Referenz auf das userMenu.

\item \textbf{ui: UI} \hfill\\ 
Eine Referenz auf die ui in der das Menu liegt. Dies wird verwendet um den Navigator aufzurufen und den Logout durchzuführen.

\end{itemize}

\underline{Methoden}
\begin{itemize}
\itemsep0pt
\item \textbf{addMenuItem (caption: String, viewId: String): void}\hfill\\
\textbf{Sichtbarkeit} public

Diese Methode dient zum Einfügen neuer Menüeinträge, die einen Ansichtswechsel auslösen sollen.

\item \textbf{addUserMenu (name: String): void}\hfill\\
\textbf{Sichtbarkeit} public

Diese Methode dient zum Einfügen eines User-Menüs. Es kann immer nur ein User-Menü geben.

\item \textbf{addLogoutItem (): void}\hfill\\
\textbf{Sichtbarkeit} public

Diese Methode dient zum Einfügen eines Logout-Eintrages. Dieser Eintrag ruft dann den Logout der Übergeordneten \nameref{UI} auf.

\end{itemize}
