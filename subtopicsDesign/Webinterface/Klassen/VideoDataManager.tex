\newpage
\subsection{VideoDataManager}\label{VideoDataManager}
Der VideoDataManager verwaltet die Video Daten und vereinfacht den Zugriff auf den \nameref{ServerProxy} für Klassen, die Video Daten benötigen.

\underline{Attribute}
\begin{itemize}
\itemsep0pt

\item \textbf{videos: LinkedList}\hfill\\
Eine Liste in dem der VideoDataManager die Videos hält die er vom ServerProxy bekommt.
\end{itemize}

\underline{Methoden}
\begin{itemize}
\itemsep0pt


\item \textbf{downloadVideo (videoId: int): void}\hfill\\
\textbf{Sichtbarkeit} public \newline
\textbf{Statisch}

Die Methode fügt die \nameref{Account} Daten hinzu und ruft danach die Methode zum downloaden am ServerProxy auf.

\item \textbf{deleteVideo (videoId: int): void}\hfill\\
\textbf{Sichtbarkeit} public \newline
\textbf{Statisch}

Die Methode fügt die Account Daten hinzu und ruft am ServerProxy die Methode zum Löschen eines Videos auf.

\item \textbf{udateVideosAndInfo (): String}\hfill\\
\textbf{Sichtbarkeit} public \newline
\textbf{Statisch}

Die Methode fügt die Account Daten hinzu und ruft am ServerProxy die Methode auf, welcher die Videos zurückgibt. Anschließend wird parseVideos aufgerufen und die Videos als Attribut gespeichert.

\item \textbf{getVideosFromServer (): String}\hfill\\
\textbf{Sichtbarkeit} private \newline
\textbf{Statisch}

Die Methode schickt eine Anfrage an den ServerProxy zum holen der Videos.

\item \textbf{getMetaInfFromServer (videoId: int): String}\hfill\\
\textbf{Sichtbarkeit} private \newline
\textbf{Statisch}

Die Methode fügt die Account Daten hinzu und ruft am ServerProxy die Methode auf, welche die Metadaten als String zurückgibt.

\item \textbf{createVideoList (videos: String, info: String): LinkedList}\hfill\\
\textbf{Sichtbarkeit} private \newline
\textbf{Statisch}

Die Methode fügt die Account Daten hinzu und ruft am ServerProxy die Methode auf, welche die Metadaten als String zurückgibt.

\end{itemize}