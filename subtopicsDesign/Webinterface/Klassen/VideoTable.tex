\newpage
\subsection{VideoTable}\label{VideoTable}
\textbf{extends}  Table \newline
Jede Zeile des VideoTables beinhaltet ein Video mit den zugehörigen Buttons zum Downloaden, Löschen und anzeigen der Infos.

\underline{Attribute}
\begin{itemize}
\itemsep0pt

\item \textbf{downLoadButtonList: LinkedList} \hfill\\ 
Eine Liste an Buttons. Zu jedem Video wird ein Button erzeugt und an die Liste gehängt.

\item \textbf{infoButtonList: LinkedList} \hfill\\ 
Eine Liste an Buttons. Zu jedem Video wird ein Button erzeugt und an die Liste gehängt.

\item \textbf{delteButtonList: LinkedList} \hfill\\ 
Eine Liste an Buttons. Zu jedem Video wird ein Button erzeugt und an die Liste gehängt.

\item \textbf{videos: LinkedList} \hfill\\ 
Das Attribut ist eine Liste der Videos. Diese wird bei Erzeugen oder Updaten vom \nameref{VideoDataManager} geholt.
\end{itemize}

\underline{Konstruktoren}
\begin{itemize}
\itemsep0pt

\item \textbf{VideoTable()} \hfill\\ 
Im Konstruktor werden die Videos über den VideoDataManager geholt. Zuden werden die Buttons vorbereitet und der Table angehängt.

\end{itemize}


\underline{Methoden}
\begin{itemize}
\itemsep0pt

\item \textbf{prepareVideos (): void}\hfill\\
\textbf{Sichtbarkeit} private

Die Videos werden zur Anzeige vorbereitet, d.h. es wird Name und Id getrennt.

\item \textbf{prepareButtons (): void}\hfill\\
\textbf{Sichtbarkeit} private

Die Buttons werden zur Anzeige vorbereitet, d.h. es werden Name und Listener gesetzt.

\item \textbf{update (): void}\hfill\\
\textbf{Sichtbarkeit} public

Dies Methode wird verwendet um den Table zu aktualisieren.

\end{itemize}
