\subsubsection{VideoTable}\label{VideoTable}
\textbf{extends}  Table \newline
Jede Zeile des VideoTables beinhaltet ein \nameref{Video} mit den zugehörigen Buttons zum Downloaden, Löschen und Anzeigen der Infos. \newline

\underline{Attribute}
\begin{itemize}
\itemsep0pt

\item \textbf{downloadButtonList: LinkedList} \hfill\\ 
\textbf{Sichtbarkeit} private

Eine Liste an Buttons. Zu jedem Video wird ein Button erzeugt und an die Liste gehängt.

\item \textbf{infoButtonList: LinkedList} \hfill\\ 
\textbf{Sichtbarkeit} private

Eine Liste an Buttons. Zu jedem Video wird ein Button erzeugt und an die Liste gehängt.

\item \textbf{deleteButtonList: LinkedList} \hfill\\ 
\textbf{Sichtbarkeit} private

Eine Liste an Buttons. Zu jedem Video wird ein Button erzeugt und an die Liste gehängt.

\item \textbf{videos: LinkedList} \hfill\\
\textbf{Sichtbarkeit} private
 
Das Attribut ist eine Liste der Videos. Die Videos werden bei Erzeugen oder Updaten des Tables vom \nameref{VideoDataManager} geholt und dann verarbeitet.
\end{itemize}

\underline{Konstruktoren}
\begin{itemize}
\itemsep0pt

\item \textbf{VideoTable()} \hfill\\ 
\textbf{Sichtbarkeit} public

Im Konstruktor werden die Videos über den VideoDataManager geholt. Zudem werden dann für jedes Video die Buttons vorbereitet.

\end{itemize}


\underline{Methoden}
\begin{itemize}
\itemsep0pt

\item \textbf{prepareVideos (): void}\hfill\\
\textbf{Sichtbarkeit} private

Die Videos werden zur Anzeige vorbereitet.

\item \textbf{prepareButtons (): void}\hfill\\
\textbf{Sichtbarkeit} private

Die Buttons werden zur Anzeige vorbereitet, d.h. es werden Name und Listener gesetzt.

\item \textbf{update (): void}\hfill\\
\textbf{Sichtbarkeit} public

Dies Methode wird verwendet um den Table zu aktualisieren.

\end{itemize}
