\newpage
\subsection{VideoTable}\label{VideoTable}
\textbf{extends}  Table \newline

Jede Zeile des VideoTables beinhaltet ein Video mit den zugehörigen Buttons zum Downloaden, Löschen und anzeigen der Infos.

\underline{Attribute}
\begin{itemize}
\itemsep0pt

\item \textbf{downLoadButtonList: LinkedList} \hfill\\ 
Eine Liste an Buttons, zu jedem Video wird ein Button erzeugt und an die Liste gehängt.

\item \textbf{infoButtonList: LinkedList} \hfill\\ 
Eine Liste an Buttons, zu jedem Video wird ein Button erzeugt und an die Liste gehängt.

\item \textbf{delteButtonList: LinkedList} \hfill\\ 
Eine Liste an Buttons, zu jedem Video wird ein Button erzeugt und an die Liste gehängt.

\item \textbf{videos: LinkedList} \hfill\\ 
Die Liste der Videos, diese wird bei erzeugen oder update vom VideoDataManager~\eqref{VideoDataManager} geholt.
\end{itemize}

\underline{Konstruktoren}
\begin{itemize}
\itemsep0pt

\item \textbf{VideoTable()} \hfill\\ 
Im Konstruktor werden die Videos über den VideoDataManager~\eqref{VideoDataManager} geholt und die Buttons vorbereitet und dann der Table entsprechend gefüllt.

\end{itemize}


\underline{Methoden}
\begin{itemize}
\itemsep0pt

\item \textbf{prepareVideos (): void}\hfill\\
\textbf{Sichtbarkeit} private

Die Videos werden zur Anzeige vorbereitet, also trennen von Name und Id.

\item \textbf{prepareButtons (): void}\hfill\\
\textbf{Sichtbarkeit} private

Die Buttons werden zur Anzeige vorbereitet, also setzten von Name und Listener.

\item \textbf{update (): void}\hfill\\
\textbf{Sichtbarkeit} public

Dies Methode wird verwendet um den Table zu aktualisieren.

\end{itemize}