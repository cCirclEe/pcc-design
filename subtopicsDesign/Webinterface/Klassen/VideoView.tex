\newpage
\subsection{VideoView}\label{VideoView}
\textbf{extends}  VerticalLayout \newline
\textbf{implements} View \newline
Diese Klasse erbt von einem Layout, da sie selbst als graphische Komponente verwendet wird. Diese View dient zum Anzeigen der Videos, die ein Benutzer mit seiner App hochgeladen hat. Zur Anzeige selbst lädt diese Klasse einen VideoTable~\eqref{VideoTable}.
\newline

\underline{Attribute}
\begin{itemize}
\itemsep0pt
\item \textbf{viewId: String} \hfill\\ 
Die viewId gibt der View eine einzigartige ID über die der Navigator die View identifizieren kann.

\end{itemize}

\underline{Konstruktoren}
\begin{itemize}
\itemsep0pt
\item \textbf{VideoView()} \hfill\\
Standardkonstruktor
\end{itemize}

\underline{Methoden}
\begin{itemize}
\itemsep0pt
\item \textbf{enter (viewChangeEvent: ViewChangeListener.ViewChangeEvent) :void}\hfill\\
\textbf{Sichtbarkeit} public

Diese Methode wird immer bei eintreten der View aufgerufen.

\item \textbf{update () :void}\hfill\\
\textbf{Sichtbarkeit} public

Diese Methode wird verwendet zum aktualisieren der View.

\end{itemize}
