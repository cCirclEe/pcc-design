\newpage
\subsection{VideoView extends VerticalLayout implements View}
Diese Klasse erbt von einem Layout, da sie selbst als graphische Komponente verwendet wird. Diese View dient zum Anzeigen der Videos, die ein Benutzer mit seiner App hochgeladen hat. Zur Anzeige selbst lädt diese Klasse einen VideoTable.

\begin{itemize}
\item \subsubsection{Attribute}
\begin{itemize}
\item \textbf{viewId String} \hfill\\ 
Die viewId gibt der View eine einzigartige ID über die der Navigator die View identifizieren kann.

\end{itemize}

\item \subsubsection{Methoden}
\begin{itemize}
\item \textbf{enter(ViewChangeListener.ViewChangeEvent viewChangeEvent)}\hfill\\
Type: public

Eingabeparameter: viewChangeEvent

Ausgabeparameter:

Diese Methode wird immer bei eintreten der View aufgerufen.



\item \textbf{update()} \hfill\\ 
Type: public

Eingabeparameter:

Ausgabeparameter:

Diese Methode wird verwendet, dass andere Klassen die Möglichkeit haben, diese View zu aktualisieren.
\end{itemize}

\end{itemize}