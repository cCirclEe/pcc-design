\subsubsection{Account} \label{service:klasse:Account}
Die Klasse Account repräsentiert den Account eines Benutzers. \newline

\underline{Attribute}
\begin{itemize}
\itemsep0pt
\item \textbf{email: String} \hfill\\ 
\textbf{Sichtbarkeit} private 
Jedem Account wird eine E-Mail-Adresse zugewiesen.

\item \textbf{passwordHash: String} \hfill\\ 
\textbf{Sichtbarkeit} private 
Jedem Account wird ein passwordHash zugewiesen.

\item \textbf{id: int} \hfill\\ 
\textbf{Sichtbarkeit} private  

Die ``id'' ist identisch der ``id'' in der Tabelle User der Datenbank.

\end{itemize}

\underline{Konstruktoren}
\begin{itemize}
\itemsep0pt
\item \textbf{Account(json: String)} \hfill\\
\textbf{Sichtbarkeit} public

Der Konstruktor nimmt ein JSON-Objekt entgegen, wertet die Informationen aus und leitet diese an die Klassenattribute weiter. Die Information ``password'' wird mit der Methode ``passwordHasher(password: String)'' gehasht und dann weitergeleitet.

\end{itemize}

\underline{Methoden}
\begin{itemize}
\itemsep0pt
\item \textbf{passwordHasher(password: String): String}\hfill\\
\textbf{Sichtbarkeit} private

Die Methode nimmt das im Klartext vorhandene Passwort und hasht dieses. Die Methode wird im Konstruktor aufgerufen, um das Passwort zu hashen und an das Klassenattribut weiterzuleiten.

\end{itemize}