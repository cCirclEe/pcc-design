\subsubsection{<<Abstract>> AAnonymizer} \label{service:klasse:AAnonymizer}
\textbf{implements} \nameref{service:klasse:IStage} \newline
IAnonymizer bietet eine Schnittstelle für Video-Anonymisierungsverfahren. Die Schnittstelle ist bewusst sehr allgemein gehalten um verschiedenste Anonymisierungsverfahren zu erlauben. \newline

\underline{Methoden}
\begin{itemize}
\itemsep0pt
\item \textbf{execute (context: Context): boolean}\hfill\\
\textbf{Sichtbarkeit} public

Implementiert die Methode execute(..) von IStage. Ruft die Methode anonymize(..) auf.

\item \textbf{<<abstract>> anonymize (input: File, output: File): boolean}\hfill\\
\textbf{Sichtbarkeit} public

Analysiert zunächst das input-Video um Bildbereiche zu identifizieren, die pesonenbezogene Daten zeigen. Wendet daraufhin einen Bildfilter auf die Bildbereiche an, um diese unkenntlich zu machen.

\end{itemize}