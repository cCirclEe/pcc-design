\subsubsection{Metadata}
Die Klasse Metadata beinhaltet alle wichigen Informationen zu den Metadaten eines Videos. \newline

\underline{Attribute}
\begin{itemize}
\itemsep0pt
\item \textbf{date: String} \hfill\\ 
\textbf{Sichtbarkeit} private 

Datum, an dem das Video aufgenommen wurde.

\item \textbf{triggerType: String} \hfill\\ 
\textbf{Sichtbarkeit} private 

Art, wie das Video ausgelöst wurde (G-Sensor, manuelle Auslösung). 

\item \textbf{gForce: Vector3D} \hfill\\ 
\textbf{Sichtbarkeit} private 

Messwerte des G-Sensors zum Auslösezeitpunkt.

\end{itemize}

\underline{Konstruktoren}
\begin{itemize}
\itemsep0pt
\item \textbf{Metadata(metaName :String, date :String, triggerType :String, gForce :Vector3D)} \hfill\\
\textbf{Sichtbarkeit} public 

Weist die Parameter den Attributen zu

\end{itemize}

\underline{Methoden}
\begin{itemize}
\itemsep0pt
\item \textbf{getAsJson(): String}\hfill\\
\textbf{Sichtbarkeit} public 

Die Methode gibt ein JSON-String zurück, welcher aus den Informationen aller Klassenattributen zusammengesetzt wird.

\end{itemize}